\documentclass[]{article}
\usepackage{lmodern}
\usepackage{amssymb,amsmath}
\usepackage{ifxetex,ifluatex}
\usepackage{fixltx2e} % provides \textsubscript
\ifnum 0\ifxetex 1\fi\ifluatex 1\fi=0 % if pdftex
  \usepackage[T1]{fontenc}
  \usepackage[utf8]{inputenc}
\else % if luatex or xelatex
  \ifxetex
    \usepackage{mathspec}
  \else
    \usepackage{fontspec}
  \fi
  \defaultfontfeatures{Ligatures=TeX,Scale=MatchLowercase}
\fi
% use upquote if available, for straight quotes in verbatim environments
\IfFileExists{upquote.sty}{\usepackage{upquote}}{}
% use microtype if available
\IfFileExists{microtype.sty}{%
\usepackage{microtype}
\UseMicrotypeSet[protrusion]{basicmath} % disable protrusion for tt fonts
}{}
\usepackage[margin=1in]{geometry}
\usepackage{hyperref}
\hypersetup{unicode=true,
            pdftitle={HDS Exercise set 3},
            pdfauthor={Shabbeer Hassan},
            pdfborder={0 0 0},
            breaklinks=true}
\urlstyle{same}  % don't use monospace font for urls
\usepackage{color}
\usepackage{fancyvrb}
\newcommand{\VerbBar}{|}
\newcommand{\VERB}{\Verb[commandchars=\\\{\}]}
\DefineVerbatimEnvironment{Highlighting}{Verbatim}{commandchars=\\\{\}}
% Add ',fontsize=\small' for more characters per line
\usepackage{framed}
\definecolor{shadecolor}{RGB}{248,248,248}
\newenvironment{Shaded}{\begin{snugshade}}{\end{snugshade}}
\newcommand{\AlertTok}[1]{\textcolor[rgb]{0.94,0.16,0.16}{#1}}
\newcommand{\AnnotationTok}[1]{\textcolor[rgb]{0.56,0.35,0.01}{\textbf{\textit{#1}}}}
\newcommand{\AttributeTok}[1]{\textcolor[rgb]{0.77,0.63,0.00}{#1}}
\newcommand{\BaseNTok}[1]{\textcolor[rgb]{0.00,0.00,0.81}{#1}}
\newcommand{\BuiltInTok}[1]{#1}
\newcommand{\CharTok}[1]{\textcolor[rgb]{0.31,0.60,0.02}{#1}}
\newcommand{\CommentTok}[1]{\textcolor[rgb]{0.56,0.35,0.01}{\textit{#1}}}
\newcommand{\CommentVarTok}[1]{\textcolor[rgb]{0.56,0.35,0.01}{\textbf{\textit{#1}}}}
\newcommand{\ConstantTok}[1]{\textcolor[rgb]{0.00,0.00,0.00}{#1}}
\newcommand{\ControlFlowTok}[1]{\textcolor[rgb]{0.13,0.29,0.53}{\textbf{#1}}}
\newcommand{\DataTypeTok}[1]{\textcolor[rgb]{0.13,0.29,0.53}{#1}}
\newcommand{\DecValTok}[1]{\textcolor[rgb]{0.00,0.00,0.81}{#1}}
\newcommand{\DocumentationTok}[1]{\textcolor[rgb]{0.56,0.35,0.01}{\textbf{\textit{#1}}}}
\newcommand{\ErrorTok}[1]{\textcolor[rgb]{0.64,0.00,0.00}{\textbf{#1}}}
\newcommand{\ExtensionTok}[1]{#1}
\newcommand{\FloatTok}[1]{\textcolor[rgb]{0.00,0.00,0.81}{#1}}
\newcommand{\FunctionTok}[1]{\textcolor[rgb]{0.00,0.00,0.00}{#1}}
\newcommand{\ImportTok}[1]{#1}
\newcommand{\InformationTok}[1]{\textcolor[rgb]{0.56,0.35,0.01}{\textbf{\textit{#1}}}}
\newcommand{\KeywordTok}[1]{\textcolor[rgb]{0.13,0.29,0.53}{\textbf{#1}}}
\newcommand{\NormalTok}[1]{#1}
\newcommand{\OperatorTok}[1]{\textcolor[rgb]{0.81,0.36,0.00}{\textbf{#1}}}
\newcommand{\OtherTok}[1]{\textcolor[rgb]{0.56,0.35,0.01}{#1}}
\newcommand{\PreprocessorTok}[1]{\textcolor[rgb]{0.56,0.35,0.01}{\textit{#1}}}
\newcommand{\RegionMarkerTok}[1]{#1}
\newcommand{\SpecialCharTok}[1]{\textcolor[rgb]{0.00,0.00,0.00}{#1}}
\newcommand{\SpecialStringTok}[1]{\textcolor[rgb]{0.31,0.60,0.02}{#1}}
\newcommand{\StringTok}[1]{\textcolor[rgb]{0.31,0.60,0.02}{#1}}
\newcommand{\VariableTok}[1]{\textcolor[rgb]{0.00,0.00,0.00}{#1}}
\newcommand{\VerbatimStringTok}[1]{\textcolor[rgb]{0.31,0.60,0.02}{#1}}
\newcommand{\WarningTok}[1]{\textcolor[rgb]{0.56,0.35,0.01}{\textbf{\textit{#1}}}}
\usepackage{longtable,booktabs}
\usepackage{graphicx,grffile}
\makeatletter
\def\maxwidth{\ifdim\Gin@nat@width>\linewidth\linewidth\else\Gin@nat@width\fi}
\def\maxheight{\ifdim\Gin@nat@height>\textheight\textheight\else\Gin@nat@height\fi}
\makeatother
% Scale images if necessary, so that they will not overflow the page
% margins by default, and it is still possible to overwrite the defaults
% using explicit options in \includegraphics[width, height, ...]{}
\setkeys{Gin}{width=\maxwidth,height=\maxheight,keepaspectratio}
\IfFileExists{parskip.sty}{%
\usepackage{parskip}
}{% else
\setlength{\parindent}{0pt}
\setlength{\parskip}{6pt plus 2pt minus 1pt}
}
\setlength{\emergencystretch}{3em}  % prevent overfull lines
\providecommand{\tightlist}{%
  \setlength{\itemsep}{0pt}\setlength{\parskip}{0pt}}
\setcounter{secnumdepth}{0}
% Redefines (sub)paragraphs to behave more like sections
\ifx\paragraph\undefined\else
\let\oldparagraph\paragraph
\renewcommand{\paragraph}[1]{\oldparagraph{#1}\mbox{}}
\fi
\ifx\subparagraph\undefined\else
\let\oldsubparagraph\subparagraph
\renewcommand{\subparagraph}[1]{\oldsubparagraph{#1}\mbox{}}
\fi

%%% Use protect on footnotes to avoid problems with footnotes in titles
\let\rmarkdownfootnote\footnote%
\def\footnote{\protect\rmarkdownfootnote}

%%% Change title format to be more compact
\usepackage{titling}

% Create subtitle command for use in maketitle
\providecommand{\subtitle}[1]{
  \posttitle{
    \begin{center}\large#1\end{center}
    }
}

\setlength{\droptitle}{-2em}

  \title{HDS Exercise set 3}
    \pretitle{\vspace{\droptitle}\centering\huge}
  \posttitle{\par}
    \author{Shabbeer Hassan}
    \preauthor{\centering\large\emph}
  \postauthor{\par}
    \date{}
    \predate{}\postdate{}
  

\begin{document}
\maketitle

\hypertarget{problem-1-solution}{%
\subsubsection{Problem 1 -- Solution}\label{problem-1-solution}}

\begin{enumerate}
\def\labelenumi{(\alph{enumi})}
\item
\end{enumerate}

\begin{Shaded}
\begin{Highlighting}[]
\KeywordTok{library}\NormalTok{(tidyverse)}
\end{Highlighting}
\end{Shaded}

\begin{verbatim}
## -- Attaching packages ----------------------------------------------------------------------------------------------------------- tidyverse 1.2.1 --
\end{verbatim}

\begin{verbatim}
## v ggplot2 3.2.1     v purrr   0.3.2
## v tibble  2.1.3     v dplyr   0.8.3
## v tidyr   1.0.0     v stringr 1.4.0
## v readr   1.3.1     v forcats 0.4.0
\end{verbatim}

\begin{verbatim}
## -- Conflicts -------------------------------------------------------------------------------------------------------------- tidyverse_conflicts() --
## x dplyr::filter() masks stats::filter()
## x dplyr::lag()    masks stats::lag()
\end{verbatim}

\begin{Shaded}
\begin{Highlighting}[]
\KeywordTok{library}\NormalTok{(ggplot2)}

\CommentTok{# Dataset}
\NormalTok{HDS_ex3 <-}\StringTok{ }\KeywordTok{read.csv}\NormalTok{(}\StringTok{"E:/Dropbox/Important_Documents/Doctoral_Work/Courses/High Dimensional Stats/2019/Week 3/HDS_ex3.txt"}\NormalTok{, }\DataTypeTok{sep=}\StringTok{" "}\NormalTok{)}

\NormalTok{train <-}\StringTok{ }\NormalTok{HDS_ex3[ }\KeywordTok{c}\NormalTok{(}\DecValTok{1}\OperatorTok{:}\DecValTok{10}\NormalTok{), ]}
\NormalTok{test <-}\StringTok{ }\NormalTok{HDS_ex3[ }\OperatorTok{-}\KeywordTok{c}\NormalTok{(}\DecValTok{1}\OperatorTok{:}\DecValTok{10}\NormalTok{), ]}

\CommentTok{### Models using TRAIN data}

\CommentTok{# 1st order LM}
\NormalTok{y1_train <-}\StringTok{ }\KeywordTok{lm}\NormalTok{(y }\OperatorTok{~}\StringTok{ }\NormalTok{x, }\DataTypeTok{data =}\NormalTok{ train)}
\NormalTok{sm_y1_train <-}\StringTok{ }\KeywordTok{summary}\NormalTok{( y1_train )}
\KeywordTok{plot}\NormalTok{(}\KeywordTok{fitted}\NormalTok{(y1_train), }\KeywordTok{residuals}\NormalTok{(y1_train)) }\CommentTok{# Fitted vs Residuals}
\end{Highlighting}
\end{Shaded}

\includegraphics{Week_3_Exercises_files/figure-latex/unnamed-chunk-2-1.pdf}

\begin{Shaded}
\begin{Highlighting}[]
\CommentTok{# 2nd order LM}
\NormalTok{y2_train <-}\StringTok{ }\KeywordTok{lm}\NormalTok{(y }\OperatorTok{~}\StringTok{ }\NormalTok{x }\OperatorTok{+}\StringTok{ }\KeywordTok{poly}\NormalTok{(x, }\DecValTok{2}\NormalTok{, }\DataTypeTok{raw =}\NormalTok{ T), }\DataTypeTok{data =}\NormalTok{ train)}
\NormalTok{sm_y2_train <-}\StringTok{ }\KeywordTok{summary}\NormalTok{( y2_train )}
\KeywordTok{plot}\NormalTok{(}\KeywordTok{fitted}\NormalTok{(y2_train), }\KeywordTok{residuals}\NormalTok{(y2_train)) }\CommentTok{# Fitted vs Residuals}
\end{Highlighting}
\end{Shaded}

\includegraphics{Week_3_Exercises_files/figure-latex/unnamed-chunk-2-2.pdf}

\begin{Shaded}
\begin{Highlighting}[]
\CommentTok{# 9th order LM}
\NormalTok{y9_train <-}\StringTok{ }\KeywordTok{lm}\NormalTok{(y }\OperatorTok{~}\StringTok{ }\NormalTok{x }\OperatorTok{+}\StringTok{ }\KeywordTok{poly}\NormalTok{(x, }\DecValTok{9}\NormalTok{, }\DataTypeTok{raw =}\NormalTok{ T), }\DataTypeTok{data =}\NormalTok{ train)}
\NormalTok{sm_y9_train <-}\StringTok{ }\KeywordTok{summary}\NormalTok{( y9_train )}
\KeywordTok{plot}\NormalTok{( }\KeywordTok{fitted}\NormalTok{(y9_train), }\KeywordTok{residuals}\NormalTok{(y9_train) ) }\CommentTok{# Fitted vs Residuals}
\end{Highlighting}
\end{Shaded}

\includegraphics{Week_3_Exercises_files/figure-latex/unnamed-chunk-2-3.pdf}

\begin{enumerate}
\def\labelenumi{(\alph{enumi})}
\setcounter{enumi}{1}
\item
\end{enumerate}

\begin{Shaded}
\begin{Highlighting}[]
\CommentTok{##### Plot training data for all models}

\CommentTok{# Adding predicted values to a dataset}
\NormalTok{pred_train <-}\StringTok{ }\KeywordTok{as.data.frame}\NormalTok{(}\KeywordTok{cbind}\NormalTok{( train}\OperatorTok{$}\NormalTok{x,}
                                   \KeywordTok{predict}\NormalTok{( y1_train ), }
                                   \KeywordTok{predict}\NormalTok{( y2_train ), }
                                   \KeywordTok{predict}\NormalTok{( y9_train ) ))}
\KeywordTok{colnames}\NormalTok{( pred_train ) <-}\StringTok{ }\KeywordTok{c}\NormalTok{( }\StringTok{"x"}\NormalTok{, }\StringTok{"1st_order_LM"}\NormalTok{, }\StringTok{"2nd_order_LM"}\NormalTok{, }\StringTok{"9th_order_LM"}\NormalTok{)}

\CommentTok{# Convert from wide to long}
\NormalTok{pred_train.df <-}\StringTok{ }\NormalTok{reshape2}\OperatorTok{::}\KeywordTok{melt}\NormalTok{(pred_train, }
                      \DataTypeTok{id =} \StringTok{"x"}\NormalTok{)}

\CommentTok{# Plotting }
\NormalTok{pred_train }\OperatorTok
\StringTok{    }\KeywordTok{gather}\NormalTok{(key,value, }\StringTok{"1st_order_LM"}\NormalTok{, }\StringTok{"2nd_order_LM"}\NormalTok{, }\StringTok{"9th_order_LM"}\NormalTok{) }\OperatorTok
\StringTok{    }\KeywordTok{ggplot}\NormalTok{(}\KeywordTok{aes}\NormalTok{(}\DataTypeTok{x=}\NormalTok{x, }\DataTypeTok{y=}\NormalTok{value, }\DataTypeTok{colour=}\NormalTok{key)) }\OperatorTok{+}
\StringTok{    }\KeywordTok{geom_line}\NormalTok{() }\OperatorTok{+}
\StringTok{  }\CommentTok{#theme with white background}
\StringTok{     }\KeywordTok{theme_bw}\NormalTok{() }\OperatorTok{+}
\StringTok{  }\CommentTok{#eliminates background, gridlines, and chart border}
\StringTok{     }\KeywordTok{theme}\NormalTok{(}
       \DataTypeTok{plot.background =} \KeywordTok{element_blank}\NormalTok{()}
\NormalTok{      ,}\DataTypeTok{panel.grid.major =} \KeywordTok{element_blank}\NormalTok{()}
\NormalTok{      ,}\DataTypeTok{panel.grid.minor =} \KeywordTok{element_blank}\NormalTok{()}
\NormalTok{      ) }\OperatorTok{+}
\StringTok{  }\CommentTok{#draws x and y axis line}
\StringTok{     }\KeywordTok{theme}\NormalTok{(}\DataTypeTok{axis.line =} \KeywordTok{element_line}\NormalTok{(}\DataTypeTok{color =} \StringTok{'black'}\NormalTok{)) }\OperatorTok{+}
\StringTok{  }\KeywordTok{scale_x_continuous}\NormalTok{( }\StringTok{"X"}\NormalTok{, }\DataTypeTok{limits =} \KeywordTok{c}\NormalTok{(}\FloatTok{0.1}\NormalTok{, }\DecValTok{1}\NormalTok{) ) }\OperatorTok{+}\StringTok{ }
\StringTok{  }\KeywordTok{scale_y_continuous}\NormalTok{( }\StringTok{"Y"}\NormalTok{, }\DataTypeTok{limits =} \KeywordTok{c}\NormalTok{(}\FloatTok{0.7}\NormalTok{, }\DecValTok{1}\NormalTok{)) }\OperatorTok{+}
\StringTok{  }\KeywordTok{labs}\NormalTok{(}\DataTypeTok{title =} \StringTok{"Predicted values from different order linear models"}\NormalTok{)}
\end{Highlighting}
\end{Shaded}

\includegraphics{Week_3_Exercises_files/figure-latex/unnamed-chunk-3-1.pdf}

\begin{enumerate}
\def\labelenumi{(\alph{enumi})}
\setcounter{enumi}{2}
\item
\end{enumerate}

\begin{Shaded}
\begin{Highlighting}[]
\CommentTok{### Models using test }

\CommentTok{# 1st order LM}
\NormalTok{y1_test <-}\StringTok{ }\KeywordTok{lm}\NormalTok{(y }\OperatorTok{~}\StringTok{ }\NormalTok{x, }\DataTypeTok{data =}\NormalTok{ test)}
\NormalTok{sm_y1_test <-}\StringTok{ }\KeywordTok{summary}\NormalTok{( y1_test )}
\KeywordTok{plot}\NormalTok{(}\KeywordTok{fitted}\NormalTok{(y1_test), }\KeywordTok{residuals}\NormalTok{(y1_test)) }\CommentTok{# Fitted vs Residuals}
\end{Highlighting}
\end{Shaded}

\includegraphics{Week_3_Exercises_files/figure-latex/unnamed-chunk-4-1.pdf}

\begin{Shaded}
\begin{Highlighting}[]
\CommentTok{# 2nd order LM}
\NormalTok{y2_test <-}\StringTok{ }\KeywordTok{lm}\NormalTok{(y }\OperatorTok{~}\StringTok{ }\NormalTok{x }\OperatorTok{+}\StringTok{ }\KeywordTok{poly}\NormalTok{(x, }\DecValTok{2}\NormalTok{, }\DataTypeTok{raw =}\NormalTok{ T), }\DataTypeTok{data =}\NormalTok{ test)}
\NormalTok{sm_y2_test <-}\StringTok{ }\KeywordTok{summary}\NormalTok{( y2_test )}
\KeywordTok{plot}\NormalTok{(}\KeywordTok{fitted}\NormalTok{(y2_test), }\KeywordTok{residuals}\NormalTok{(y2_test)) }\CommentTok{# Fitted vs Residuals}
\end{Highlighting}
\end{Shaded}

\includegraphics{Week_3_Exercises_files/figure-latex/unnamed-chunk-4-2.pdf}

\begin{Shaded}
\begin{Highlighting}[]
\CommentTok{# 9th order LM}
\NormalTok{y9_test <-}\StringTok{ }\KeywordTok{lm}\NormalTok{(y }\OperatorTok{~}\StringTok{ }\NormalTok{x }\OperatorTok{+}\StringTok{ }\KeywordTok{poly}\NormalTok{(x, }\DecValTok{9}\NormalTok{, }\DataTypeTok{raw =}\NormalTok{ T), }\DataTypeTok{data =}\NormalTok{ test)}
\NormalTok{sm_y9_test <-}\StringTok{ }\KeywordTok{summary}\NormalTok{( y9_test )}
\KeywordTok{plot}\NormalTok{( }\KeywordTok{fitted}\NormalTok{(y9_test), }\KeywordTok{residuals}\NormalTok{(y9_test) ) }\CommentTok{# Fitted vs Residuals}
\end{Highlighting}
\end{Shaded}

\includegraphics{Week_3_Exercises_files/figure-latex/unnamed-chunk-4-3.pdf}

\begin{Shaded}
\begin{Highlighting}[]
\CommentTok{# MSE function}
\NormalTok{mse <-}\StringTok{ }\ControlFlowTok{function}\NormalTok{(sm) }
    \KeywordTok{mean}\NormalTok{(sm}\OperatorTok{$}\NormalTok{residuals}\OperatorTok{^}\DecValTok{2}\NormalTok{)}

\CommentTok{#mse <- function(sm) \{sum( sm$residuals^2 )/sm$df.residual\}}

\CommentTok{# Obtain }\AlertTok{TEST}\CommentTok{ MSE's}
\NormalTok{y1_test_mse <-}\StringTok{ }\KeywordTok{mse}\NormalTok{(sm_y1_test)}
\NormalTok{y2_test_mse <-}\StringTok{ }\KeywordTok{mse}\NormalTok{(sm_y2_test)}
\NormalTok{y9_test_mse <-}\StringTok{ }\KeywordTok{mse}\NormalTok{(sm_y9_test)}
\NormalTok{test_mse.df <-}\StringTok{ }\KeywordTok{as.data.frame}\NormalTok{(}\KeywordTok{cbind}\NormalTok{( }\StringTok{"Test_MSE"}\NormalTok{, y1_test_mse, y2_test_mse, y9_test_mse ))}
\KeywordTok{colnames}\NormalTok{(test_mse.df) <-}\StringTok{ }\KeywordTok{c}\NormalTok{(}\StringTok{"Data"}\NormalTok{, }\StringTok{"1st_order"}\NormalTok{, }\StringTok{"2nd order"}\NormalTok{, }\StringTok{"9th_order"}\NormalTok{)}

\CommentTok{# Obtain TRAIN MSE's}
\NormalTok{y1_train_mse <-}\StringTok{ }\KeywordTok{mse}\NormalTok{(sm_y2_train)}
\NormalTok{y2_train_mse <-}\StringTok{ }\KeywordTok{mse}\NormalTok{(sm_y2_train)}
\NormalTok{y9_train_mse <-}\StringTok{ }\KeywordTok{mse}\NormalTok{(sm_y9_train)}
\NormalTok{train_mse.df <-}\StringTok{ }\KeywordTok{as.data.frame}\NormalTok{(}\KeywordTok{cbind}\NormalTok{( }\StringTok{"Train_MSE"}\NormalTok{, y1_train_mse, y2_train_mse, y9_train_mse ))}
\KeywordTok{colnames}\NormalTok{(train_mse.df) <-}\StringTok{ }\KeywordTok{c}\NormalTok{(}\StringTok{"Data"}\NormalTok{, }\StringTok{"1st_order"}\NormalTok{, }\StringTok{"2nd order"}\NormalTok{, }\StringTok{"9th_order"}\NormalTok{)}

\CommentTok{# Get a df to showcase diff bet train and test MSE}
\NormalTok{diff_mse <-}\StringTok{ }\KeywordTok{rbind}\NormalTok{(test_mse.df, train_mse.df)}
\NormalTok{diff_mse}
\end{Highlighting}
\end{Shaded}

\begin{verbatim}
##        Data            1st_order            2nd order            9th_order
## 1  Test_MSE   0.0083714171301726  0.00273715162189185  0.00256754067814762
## 2 Train_MSE 0.000838066404415258 0.000838066404415258 2.09569420646987e-05
\end{verbatim}

\hypertarget{training-mses-for-the-1st-2nd-and-9th-order-models-are-higher-than-the-test-mses.-in-fact-as-the-order-of-model-increases-the-mse-of-test-dataset-remains-similar-but-the-training-set-mses-decrease.}{%
\paragraph{Training MSE's for the 1st, 2nd and 9th order models are
higher than the test MSE's. In fact, as the order of model increases,
the MSE of test dataset remains similar but the training set MSE's
decrease.}\label{training-mses-for-the-1st-2nd-and-9th-order-models-are-higher-than-the-test-mses.-in-fact-as-the-order-of-model-increases-the-mse-of-test-dataset-remains-similar-but-the-training-set-mses-decrease.}}

\begin{enumerate}
\def\labelenumi{(\alph{enumi})}
\setcounter{enumi}{3}
\item
\end{enumerate}

\begin{Shaded}
\begin{Highlighting}[]
\CommentTok{### Plot predicted x-values against actual x from train data}

\CommentTok{## Prediction for each models}
\NormalTok{predict_y1 <-}\StringTok{ }\KeywordTok{predict}\NormalTok{(y1_train, }\KeywordTok{data.frame}\NormalTok{(}\DataTypeTok{x =}\NormalTok{ test}\OperatorTok{$}\NormalTok{x), }
                      \DataTypeTok{interval =} \StringTok{'confidence'}\NormalTok{,}
                      \DataTypeTok{level =} \FloatTok{0.99}\NormalTok{)}

\NormalTok{predict_y2 <-}\StringTok{ }\KeywordTok{predict}\NormalTok{(y2_train, }\KeywordTok{data.frame}\NormalTok{(}\DataTypeTok{x =}\NormalTok{ test}\OperatorTok{$}\NormalTok{x), }
                      \DataTypeTok{interval =} \StringTok{'confidence'}\NormalTok{,}
                      \DataTypeTok{level =} \FloatTok{0.99}\NormalTok{)}
\end{Highlighting}
\end{Shaded}

\begin{verbatim}
## Warning in predict.lm(y2_train, data.frame(x = test$x), interval =
## "confidence", : prediction from a rank-deficient fit may be misleading
\end{verbatim}

\begin{Shaded}
\begin{Highlighting}[]
\NormalTok{predict_y9 <-}\StringTok{ }\KeywordTok{predict}\NormalTok{(y9_train, }\KeywordTok{data.frame}\NormalTok{(}\DataTypeTok{x =}\NormalTok{ test}\OperatorTok{$}\NormalTok{x), }
                      \DataTypeTok{interval =} \StringTok{'confidence'}\NormalTok{,}
                      \DataTypeTok{level =} \FloatTok{0.99}\NormalTok{)}
\end{Highlighting}
\end{Shaded}

\begin{verbatim}
## Warning in predict.lm(y9_train, data.frame(x = test$x), interval =
## "confidence", : prediction from a rank-deficient fit may be misleading
\end{verbatim}

\begin{Shaded}
\begin{Highlighting}[]
\CommentTok{# Adding all predicted values in a df}
\NormalTok{predicted_df <-}\StringTok{ }\KeywordTok{as.data.frame}\NormalTok{(}\KeywordTok{cbind}\NormalTok{( test}\OperatorTok{$}\NormalTok{x, predict_y1[,}\DecValTok{1}\NormalTok{], predict_y2[,}\DecValTok{1}\NormalTok{], predict_y9[,}\DecValTok{1}\NormalTok{] ))}
\KeywordTok{colnames}\NormalTok{(predicted_df) <-}\StringTok{ }\KeywordTok{c}\NormalTok{( }\StringTok{"x"}\NormalTok{, }\StringTok{"1st_order_LM"}\NormalTok{, }\StringTok{"2nd_order_LM"}\NormalTok{, }\StringTok{"9th_order_LM"}\NormalTok{)}


\CommentTok{# Convert from wide to long}
\NormalTok{predicted_test.df <-}\StringTok{ }\NormalTok{reshape2}\OperatorTok{::}\KeywordTok{melt}\NormalTok{(predicted_df, }
                      \DataTypeTok{id =} \StringTok{"x"}\NormalTok{)}

\CommentTok{# Plotting }
\NormalTok{predicted_df }\OperatorTok
\StringTok{     }\KeywordTok{gather}\NormalTok{(key,value, }\StringTok{"1st_order_LM"}\NormalTok{, }\StringTok{"2nd_order_LM"}\NormalTok{, }\StringTok{"9th_order_LM"}\NormalTok{) }\OperatorTok
\StringTok{    }\KeywordTok{ggplot}\NormalTok{(}\KeywordTok{aes}\NormalTok{(}\DataTypeTok{x=}\NormalTok{x, }\DataTypeTok{y=}\NormalTok{value, }\DataTypeTok{colour=}\NormalTok{key)) }\OperatorTok{+}
\StringTok{    }\KeywordTok{geom_line}\NormalTok{() }\OperatorTok{+}
\StringTok{  }\CommentTok{#theme with white background}
\StringTok{     }\KeywordTok{theme_bw}\NormalTok{() }\OperatorTok{+}
\StringTok{  }\CommentTok{#eliminates background, gridlines, and chart border}
\StringTok{     }\KeywordTok{theme}\NormalTok{(}
       \DataTypeTok{plot.background =} \KeywordTok{element_blank}\NormalTok{()}
\NormalTok{      ,}\DataTypeTok{panel.grid.major =} \KeywordTok{element_blank}\NormalTok{()}
\NormalTok{      ,}\DataTypeTok{panel.grid.minor =} \KeywordTok{element_blank}\NormalTok{()}
\NormalTok{      ) }\OperatorTok{+}
\StringTok{  }\CommentTok{#draws x and y axis line}
\StringTok{     }\KeywordTok{theme}\NormalTok{(}\DataTypeTok{axis.line =} \KeywordTok{element_line}\NormalTok{(}\DataTypeTok{color =} \StringTok{'black'}\NormalTok{)) }\OperatorTok{+}
\StringTok{  }\KeywordTok{scale_x_continuous}\NormalTok{( }\StringTok{"X"}\NormalTok{, }\DataTypeTok{limits =} \KeywordTok{c}\NormalTok{(}\DecValTok{0}\NormalTok{, }\FloatTok{1.1}\NormalTok{) ) }\OperatorTok{+}\StringTok{ }
\StringTok{  }\KeywordTok{scale_y_continuous}\NormalTok{( }\StringTok{"Y"}\NormalTok{, }\DataTypeTok{limits =} \KeywordTok{c}\NormalTok{(}\DecValTok{0}\NormalTok{, }\FloatTok{2.1}\NormalTok{)) }\OperatorTok{+}
\StringTok{  }\KeywordTok{labs}\NormalTok{(}\DataTypeTok{title =} \StringTok{"Predicted values from different order linear models"}\NormalTok{)}
\end{Highlighting}
\end{Shaded}

\begin{verbatim}
## Warning: Removed 18 rows containing missing values (geom_path).
\end{verbatim}

\includegraphics{Week_3_Exercises_files/figure-latex/unnamed-chunk-5-1.pdf}

\hypertarget{the-first-degree-model-is-reasonable-but-we-can-see-that-the-second-degree-model-fits-much-better.-the-ninth-degree-model-seem-rather-wild.}{%
\paragraph{The first degree model is reasonable, but we can see that the
second degree model fits much better. The ninth degree model seem rather
wild.}\label{the-first-degree-model-is-reasonable-but-we-can-see-that-the-second-degree-model-fits-much-better.-the-ninth-degree-model-seem-rather-wild.}}

(e).

\begin{Shaded}
\begin{Highlighting}[]
\CommentTok{# Get truth "y" into predicted dataframe}
\NormalTok{pred_truth.df <-}\StringTok{ }\KeywordTok{as.data.frame}\NormalTok{( }\KeywordTok{cbind}\NormalTok{(test}\OperatorTok{$}\NormalTok{y, predicted_df) )}
\KeywordTok{colnames}\NormalTok{(pred_truth.df) <-}\StringTok{ }\KeywordTok{c}\NormalTok{(}\StringTok{"truth"}\NormalTok{, }\StringTok{"x"}\NormalTok{, }\StringTok{"1st_order_LM"}\NormalTok{, }\StringTok{"2nd_order_LM"}\NormalTok{, }\StringTok{"9th_order_LM"}\NormalTok{)}

\CommentTok{## Bias-Variance tradeoff}
\NormalTok{get_bias =}\StringTok{ }\ControlFlowTok{function}\NormalTok{(estimate, truth) \{}
  \KeywordTok{mean}\NormalTok{(estimate) }\OperatorTok{-}\StringTok{ }\NormalTok{truth}
\NormalTok{\}}

\NormalTok{get_var =}\StringTok{ }\ControlFlowTok{function}\NormalTok{(estimate) \{}
  \KeywordTok{mean}\NormalTok{((estimate }\OperatorTok{-}\StringTok{ }\KeywordTok{mean}\NormalTok{(estimate)) }\OperatorTok{^}\StringTok{ }\DecValTok{2}\NormalTok{)}
\NormalTok{\}}

\NormalTok{get_mse =}\StringTok{ }\ControlFlowTok{function}\NormalTok{(truth, estimate) \{}
  \KeywordTok{mean}\NormalTok{((estimate }\OperatorTok{-}\StringTok{ }\NormalTok{truth) }\OperatorTok{^}\StringTok{ }\DecValTok{2}\NormalTok{)}
\NormalTok{\}}


\CommentTok{# Bias from 3 models}
\NormalTok{bias_1st <-}\StringTok{ }\KeywordTok{get_bias}\NormalTok{(pred_truth.df}\OperatorTok{$}\StringTok{`}\DataTypeTok{1st_order_LM}\StringTok{`}\NormalTok{, pred_truth.df}\OperatorTok{$}\NormalTok{truth)}
\NormalTok{bias_2nd <-}\StringTok{ }\KeywordTok{get_bias}\NormalTok{(pred_truth.df}\OperatorTok{$}\StringTok{`}\DataTypeTok{2nd_order_LM}\StringTok{`}\NormalTok{, pred_truth.df}\OperatorTok{$}\NormalTok{truth)}
\NormalTok{bias_9th <-}\StringTok{ }\KeywordTok{get_bias}\NormalTok{(pred_truth.df}\OperatorTok{$}\StringTok{`}\DataTypeTok{9th_order_LM}\StringTok{`}\NormalTok{, pred_truth.df}\OperatorTok{$}\NormalTok{truth)}

\NormalTok{bias_df <-}\StringTok{ }\KeywordTok{as.data.frame}\NormalTok{( }\KeywordTok{cbind}\NormalTok{(bias_1st, bias_2nd, bias_9th) )}
\NormalTok{bias <-}\StringTok{ }\KeywordTok{c}\NormalTok{(}\KeywordTok{mean}\NormalTok{(bias_df}\OperatorTok{$}\NormalTok{bias_1st), }\KeywordTok{mean}\NormalTok{(bias_df}\OperatorTok{$}\NormalTok{bias_2nd), }\KeywordTok{mean}\NormalTok{(bias_df}\OperatorTok{$}\NormalTok{bias_9th))}

\CommentTok{# Variance from 3 models}
\NormalTok{var_1st <-}\StringTok{ }\KeywordTok{get_var}\NormalTok{(pred_truth.df}\OperatorTok{$}\StringTok{`}\DataTypeTok{1st_order_LM}\StringTok{`}\NormalTok{)}
\NormalTok{var_2nd <-}\StringTok{ }\KeywordTok{get_var}\NormalTok{(pred_truth.df}\OperatorTok{$}\StringTok{`}\DataTypeTok{2nd_order_LM}\StringTok{`}\NormalTok{)}
\NormalTok{var_9th <-}\StringTok{ }\KeywordTok{get_var}\NormalTok{(pred_truth.df}\OperatorTok{$}\StringTok{`}\DataTypeTok{9th_order_LM}\StringTok{`}\NormalTok{)}

\NormalTok{var <-}\StringTok{ }\KeywordTok{as.data.frame}\NormalTok{( }\KeywordTok{rbind}\NormalTok{(var_1st, var_2nd, var_9th) )}

\CommentTok{# MSE from 3 models}
\NormalTok{mse_1st <-}\StringTok{ }\KeywordTok{get_mse}\NormalTok{(pred_truth.df}\OperatorTok{$}\StringTok{`}\DataTypeTok{1st_order_LM}\StringTok{`}\NormalTok{, pred_truth.df}\OperatorTok{$}\NormalTok{truth)}
\NormalTok{mse_2nd <-}\StringTok{ }\KeywordTok{get_mse}\NormalTok{(pred_truth.df}\OperatorTok{$}\StringTok{`}\DataTypeTok{2nd_order_LM}\StringTok{`}\NormalTok{, pred_truth.df}\OperatorTok{$}\NormalTok{truth)}
\NormalTok{mse_9th <-}\StringTok{ }\KeywordTok{get_mse}\NormalTok{(pred_truth.df}\OperatorTok{$}\StringTok{`}\DataTypeTok{9th_order_LM}\StringTok{`}\NormalTok{, pred_truth.df}\OperatorTok{$}\NormalTok{truth)}

\NormalTok{mse <-}\StringTok{ }\KeywordTok{as.data.frame}\NormalTok{( }\KeywordTok{rbind}\NormalTok{(mse_1st, mse_2nd, mse_9th) )}

\CommentTok{# Summarize these above results in the following table}
\NormalTok{results <-}\StringTok{ }\KeywordTok{data.frame}\NormalTok{ (}
  \KeywordTok{cbind}\NormalTok{(}\DataTypeTok{poly_degree =} \KeywordTok{c}\NormalTok{(}\DecValTok{1}\NormalTok{, }\DecValTok{2}\NormalTok{, }\DecValTok{9}\NormalTok{),}
        \KeywordTok{round}\NormalTok{(bias}\OperatorTok{^}\DecValTok{2}\NormalTok{, }\DecValTok{5}\NormalTok{),}
        \KeywordTok{round}\NormalTok{(mse, }\DecValTok{5}\NormalTok{),}
        \KeywordTok{round}\NormalTok{(var, }\DecValTok{5}\NormalTok{))}
\NormalTok{)}
\KeywordTok{colnames}\NormalTok{(results) =}\StringTok{ }\KeywordTok{c}\NormalTok{(}\StringTok{"Degree"}\NormalTok{, }\StringTok{"Mean Squared Error"}\NormalTok{, }\StringTok{"Bias Squared"}\NormalTok{, }\StringTok{"Variance"}\NormalTok{)}
\KeywordTok{rownames}\NormalTok{(results) =}\StringTok{ }\OtherTok{NULL}
\NormalTok{knitr}\OperatorTok{::}\KeywordTok{kable}\NormalTok{(results, }\DataTypeTok{booktabs =} \OtherTok{TRUE}\NormalTok{, }\DataTypeTok{escape =} \OtherTok{TRUE}\NormalTok{, }\DataTypeTok{align =} \StringTok{"c"}\NormalTok{)}
\end{Highlighting}
\end{Shaded}

\begin{longtable}[]{@{}cccc@{}}
\toprule
Degree & Mean Squared Error & Bias Squared & Variance\tabularnewline
\midrule
\endhead
1 & 0.00006 & 0.01160 & 0.00158\tabularnewline
2 & 0.00028 & 0.00305 & 0.00650\tabularnewline
9 & 25.35862 & 270.84226 & 246.74820\tabularnewline
\bottomrule
\end{longtable}

\begin{Shaded}
\begin{Highlighting}[]
\CommentTok{# Bias-Variance Tradeoff}
\CommentTok{# Defined as, bias ^ 2 + variance == mse}
\end{Highlighting}
\end{Shaded}

\hypertarget{we-see-that-the-bias-in-general-decreases-upon-considering-the-2nd-order-model-than-1st-one}{%
\paragraph{We see that the Bias in general decreases upon considering
the 2nd order model than 1st
one}\label{we-see-that-the-bias-in-general-decreases-upon-considering-the-2nd-order-model-than-1st-one}}

\hypertarget{we-see-that-the-2nd-order-model-gets-the-best-bias-variance-tradeoff-here}{%
\subparagraph{We see that the 2nd order model gets the best
bias-variance tradeoff
here}\label{we-see-that-the-2nd-order-model-gets-the-best-bias-variance-tradeoff-here}}

\hypertarget{problem-2-solution}{%
\subsubsection{Problem 2 -- Solution}\label{problem-2-solution}}

\begin{enumerate}
\def\labelenumi{(\alph{enumi})}
\item
\end{enumerate}

\begin{Shaded}
\begin{Highlighting}[]
\KeywordTok{library}\NormalTok{(MASS)}
\end{Highlighting}
\end{Shaded}

\begin{verbatim}
## 
## Attaching package: 'MASS'
\end{verbatim}

\begin{verbatim}
## The following object is masked from 'package:dplyr':
## 
##     select
\end{verbatim}

\begin{Shaded}
\begin{Highlighting}[]
\KeywordTok{data}\NormalTok{(Boston)}

\CommentTok{#Using all variables as predictors}

\NormalTok{lm.fit <-}\StringTok{ }\KeywordTok{lm}\NormalTok{(medv }\OperatorTok{~}\StringTok{ }\NormalTok{., Boston)}
\KeywordTok{summary}\NormalTok{(lm.fit)}
\end{Highlighting}
\end{Shaded}

\begin{verbatim}
## 
## Call:
## lm(formula = medv ~ ., data = Boston)
## 
## Residuals:
##     Min      1Q  Median      3Q     Max 
## -15.595  -2.730  -0.518   1.777  26.199 
## 
## Coefficients:
##               Estimate Std. Error t value Pr(>|t|)    
## (Intercept)  3.646e+01  5.103e+00   7.144 3.28e-12 ***
## crim        -1.080e-01  3.286e-02  -3.287 0.001087 ** 
## zn           4.642e-02  1.373e-02   3.382 0.000778 ***
## indus        2.056e-02  6.150e-02   0.334 0.738288    
## chas         2.687e+00  8.616e-01   3.118 0.001925 ** 
## nox         -1.777e+01  3.820e+00  -4.651 4.25e-06 ***
## rm           3.810e+00  4.179e-01   9.116  < 2e-16 ***
## age          6.922e-04  1.321e-02   0.052 0.958229    
## dis         -1.476e+00  1.995e-01  -7.398 6.01e-13 ***
## rad          3.060e-01  6.635e-02   4.613 5.07e-06 ***
## tax         -1.233e-02  3.760e-03  -3.280 0.001112 ** 
## ptratio     -9.527e-01  1.308e-01  -7.283 1.31e-12 ***
## black        9.312e-03  2.686e-03   3.467 0.000573 ***
## lstat       -5.248e-01  5.072e-02 -10.347  < 2e-16 ***
## ---
## Signif. codes:  0 '***' 0.001 '**' 0.01 '*' 0.05 '.' 0.1 ' ' 1
## 
## Residual standard error: 4.745 on 492 degrees of freedom
## Multiple R-squared:  0.7406, Adjusted R-squared:  0.7338 
## F-statistic: 108.1 on 13 and 492 DF,  p-value: < 2.2e-16
\end{verbatim}

\begin{Shaded}
\begin{Highlighting}[]
\KeywordTok{AIC}\NormalTok{(lm.fit)}
\end{Highlighting}
\end{Shaded}

\begin{verbatim}
## [1] 3027.609
\end{verbatim}

\begin{Shaded}
\begin{Highlighting}[]
\KeywordTok{BIC}\NormalTok{(lm.fit)}
\end{Highlighting}
\end{Shaded}

\begin{verbatim}
## [1] 3091.007
\end{verbatim}

\begin{Shaded}
\begin{Highlighting}[]
\CommentTok{# Using all but age}
\NormalTok{lm.fit1 <-}\StringTok{ }\KeywordTok{lm}\NormalTok{(medv∼.}\OperatorTok{-}\NormalTok{age ,}\DataTypeTok{data=}\NormalTok{Boston )}
\KeywordTok{summary}\NormalTok{ (lm.fit1)}
\end{Highlighting}
\end{Shaded}

\begin{verbatim}
## 
## Call:
## lm(formula = medv ~ . - age, data = Boston)
## 
## Residuals:
##      Min       1Q   Median       3Q      Max 
## -15.6054  -2.7313  -0.5188   1.7601  26.2243 
## 
## Coefficients:
##               Estimate Std. Error t value Pr(>|t|)    
## (Intercept)  36.436927   5.080119   7.172 2.72e-12 ***
## crim         -0.108006   0.032832  -3.290 0.001075 ** 
## zn            0.046334   0.013613   3.404 0.000719 ***
## indus         0.020562   0.061433   0.335 0.737989    
## chas          2.689026   0.859598   3.128 0.001863 ** 
## nox         -17.713540   3.679308  -4.814 1.97e-06 ***
## rm            3.814394   0.408480   9.338  < 2e-16 ***
## dis          -1.478612   0.190611  -7.757 5.03e-14 ***
## rad           0.305786   0.066089   4.627 4.75e-06 ***
## tax          -0.012329   0.003755  -3.283 0.001099 ** 
## ptratio      -0.952211   0.130294  -7.308 1.10e-12 ***
## black         0.009321   0.002678   3.481 0.000544 ***
## lstat        -0.523852   0.047625 -10.999  < 2e-16 ***
## ---
## Signif. codes:  0 '***' 0.001 '**' 0.01 '*' 0.05 '.' 0.1 ' ' 1
## 
## Residual standard error: 4.74 on 493 degrees of freedom
## Multiple R-squared:  0.7406, Adjusted R-squared:  0.7343 
## F-statistic: 117.3 on 12 and 493 DF,  p-value: < 2.2e-16
\end{verbatim}

\begin{Shaded}
\begin{Highlighting}[]
\KeywordTok{AIC}\NormalTok{(lm.fit1)}
\end{Highlighting}
\end{Shaded}

\begin{verbatim}
## [1] 3025.611
\end{verbatim}

\begin{Shaded}
\begin{Highlighting}[]
\KeywordTok{BIC}\NormalTok{(lm.fit1)}
\end{Highlighting}
\end{Shaded}

\begin{verbatim}
## [1] 3084.783
\end{verbatim}

\begin{Shaded}
\begin{Highlighting}[]
\CommentTok{# Using all but rm}
\NormalTok{lm.fit2 <-}\StringTok{ }\KeywordTok{lm}\NormalTok{(medv∼.}\OperatorTok{-}\NormalTok{rm ,}\DataTypeTok{data=}\NormalTok{Boston )}
\KeywordTok{summary}\NormalTok{ (lm.fit2)}
\end{Highlighting}
\end{Shaded}

\begin{verbatim}
## 
## Call:
## lm(formula = medv ~ . - rm, data = Boston)
## 
## Residuals:
##      Min       1Q   Median       3Q      Max 
## -11.8415  -2.9471  -0.5922   1.7921  23.1236 
## 
## Coefficients:
##               Estimate Std. Error t value Pr(>|t|)    
## (Intercept)  69.467163   3.884501  17.883  < 2e-16 ***
## crim         -0.115899   0.035484  -3.266 0.001166 ** 
## zn            0.065917   0.014645   4.501 8.45e-06 ***
## indus        -0.030978   0.066138  -0.468 0.639713    
## chas          2.931577   0.930110   3.152 0.001721 ** 
## nox         -21.021201   4.107510  -5.118 4.44e-07 ***
## age           0.025592   0.013959   1.833 0.067351 .  
## dis          -1.718414   0.213494  -8.049 6.29e-15 ***
## rad           0.401657   0.070757   5.677 2.35e-08 ***
## tax          -0.014881   0.004050  -3.674 0.000265 ***
## ptratio      -1.142943   0.139493  -8.194 2.20e-15 ***
## black         0.006747   0.002885   2.339 0.019752 *  
## lstat        -0.772070   0.046280 -16.683  < 2e-16 ***
## ---
## Signif. codes:  0 '***' 0.001 '**' 0.01 '*' 0.05 '.' 0.1 ' ' 1
## 
## Residual standard error: 5.125 on 493 degrees of freedom
## Multiple R-squared:  0.6968, Adjusted R-squared:  0.6895 
## F-statistic: 94.43 on 12 and 493 DF,  p-value: < 2.2e-16
\end{verbatim}

\begin{Shaded}
\begin{Highlighting}[]
\KeywordTok{AIC}\NormalTok{(lm.fit2)}
\end{Highlighting}
\end{Shaded}

\begin{verbatim}
## [1] 3104.581
\end{verbatim}

\begin{Shaded}
\begin{Highlighting}[]
\KeywordTok{BIC}\NormalTok{(lm.fit2)}
\end{Highlighting}
\end{Shaded}

\begin{verbatim}
## [1] 3163.753
\end{verbatim}

\begin{Shaded}
\begin{Highlighting}[]
\CommentTok{### Comparing AIC & BIC}
\KeywordTok{AIC}\NormalTok{(lm.fit, lm.fit1, lm.fit2)}
\end{Highlighting}
\end{Shaded}

\begin{verbatim}
##         df      AIC
## lm.fit  15 3027.609
## lm.fit1 14 3025.611
## lm.fit2 14 3104.581
\end{verbatim}

\begin{Shaded}
\begin{Highlighting}[]
\KeywordTok{BIC}\NormalTok{(lm.fit, lm.fit1, lm.fit2)}
\end{Highlighting}
\end{Shaded}

\begin{verbatim}
##         df      BIC
## lm.fit  15 3091.007
## lm.fit1 14 3084.783
## lm.fit2 14 3163.753
\end{verbatim}

\hypertarget{both-aic-and-bic-are-minimized-in-the-second-model-with-just-the-age-removed.}{%
\paragraph{Both AIC and BIC are minimized in the second model with just
the age
removed.}\label{both-aic-and-bic-are-minimized-in-the-second-model-with-just-the-age-removed.}}

\begin{enumerate}
\def\labelenumi{(\alph{enumi})}
\setcounter{enumi}{1}
\item
\end{enumerate}

\begin{Shaded}
\begin{Highlighting}[]
\CommentTok{# Get log-lk values from AIC}
\NormalTok{loglk_aic_full <-}\StringTok{ }\OperatorTok{-}\NormalTok{(}\KeywordTok{AIC}\NormalTok{(lm.fit) }\OperatorTok{-}\StringTok{ }\DecValTok{2}\OperatorTok{*}\DecValTok{15}\NormalTok{)}
\NormalTok{loglk_aic_reduced <-}\StringTok{ }\OperatorTok{-}\NormalTok{(}\KeywordTok{AIC}\NormalTok{(lm.fit1) }\OperatorTok{-}\StringTok{ }\DecValTok{2}\OperatorTok{*}\DecValTok{14}\NormalTok{)}

\NormalTok{loglk_aic_full}
\end{Highlighting}
\end{Shaded}

\begin{verbatim}
## [1] -2997.609
\end{verbatim}

\begin{Shaded}
\begin{Highlighting}[]
\NormalTok{loglk_aic_reduced}
\end{Highlighting}
\end{Shaded}

\begin{verbatim}
## [1] -2997.611
\end{verbatim}

\begin{Shaded}
\begin{Highlighting}[]
\CommentTok{# The Likelihood ratio should be converted to -2*difference in log likelihoods: −2ln(Likelihood_reduced/Likelihood_full)}
\CommentTok{# And the above can be written as:}
\CommentTok{# −2ln(Likelihood_reduced)− −2ln(Likelihood_full)}
\CommentTok{# LRT then should be approximately Chi-squared distributed with df equal to the number of fixed dimensions (difference in free parameters between the full and reduced model).}

\CommentTok{# Likelihood-Ratio test (frequentist)}
\NormalTok{Deviance <-}\StringTok{ }\NormalTok{loglk_aic_reduced }\OperatorTok{-}\StringTok{ }\NormalTok{loglk_aic_full }
\NormalTok{Deviance}
\end{Highlighting}
\end{Shaded}

\begin{verbatim}
## [1] -0.002824145
\end{verbatim}

\begin{Shaded}
\begin{Highlighting}[]
\NormalTok{Chisq.crit <-}\StringTok{ }\KeywordTok{qchisq}\NormalTok{(}\FloatTok{0.95}\NormalTok{,}\DecValTok{1}\NormalTok{)}
\NormalTok{Chisq.crit}
\end{Highlighting}
\end{Shaded}

\begin{verbatim}
## [1] 3.841459
\end{verbatim}

\begin{Shaded}
\begin{Highlighting}[]
\CommentTok{# LRT}
\NormalTok{Deviance }\OperatorTok{>=}\StringTok{ }\NormalTok{Chisq.crit   }\CommentTok{# perform the LRT}
\end{Highlighting}
\end{Shaded}

\begin{verbatim}
## [1] FALSE
\end{verbatim}

\begin{Shaded}
\begin{Highlighting}[]
\CommentTok{# p-value}
\DecValTok{1}\OperatorTok{-}\KeywordTok{pchisq}\NormalTok{(Deviance,}\DecValTok{1}\NormalTok{)}
\end{Highlighting}
\end{Shaded}

\begin{verbatim}
## [1] 1
\end{verbatim}

\hypertarget{the-p-values-for-let-calculation-based-on-aic-agrees-with-the-one-from-full-model-for-variable-age-0.95-vs-1}{%
\paragraph{The p-values for LET calculation based on AIC agrees with the
one from full model for variable age (0.95 vs
1)}\label{the-p-values-for-let-calculation-based-on-aic-agrees-with-the-one-from-full-model-for-variable-age-0.95-vs-1}}

\begin{enumerate}
\def\labelenumi{(\alph{enumi})}
\setcounter{enumi}{2}
\item
\end{enumerate}

\begin{Shaded}
\begin{Highlighting}[]
\CommentTok{# Get log-lk values from BIC}
\NormalTok{loglk_bic_full <-}\StringTok{ }\OperatorTok{-}\NormalTok{(}\KeywordTok{BIC}\NormalTok{(lm.fit) }\OperatorTok{-}\StringTok{ }\DecValTok{2}\OperatorTok{*}\DecValTok{15}\NormalTok{)}
\NormalTok{loglk_bic_reduced <-}\StringTok{ }\OperatorTok{-}\NormalTok{(}\KeywordTok{BIC}\NormalTok{(lm.fit2) }\OperatorTok{-}\StringTok{ }\DecValTok{2}\OperatorTok{*}\DecValTok{14}\NormalTok{)}

\NormalTok{loglk_bic_full}
\end{Highlighting}
\end{Shaded}

\begin{verbatim}
## [1] -3061.007
\end{verbatim}

\begin{Shaded}
\begin{Highlighting}[]
\NormalTok{loglk_bic_reduced}
\end{Highlighting}
\end{Shaded}

\begin{verbatim}
## [1] -3135.753
\end{verbatim}

\begin{Shaded}
\begin{Highlighting}[]
\CommentTok{# The Likelihood ratio should be converted to -2*difference in log likelihoods: −2ln(Likelihood_reduced/Likelihood_full)}
\CommentTok{# And the above can be written as:}
\CommentTok{# −2ln(Likelihood_reduced)− −2ln(Likelihood_full)}
\CommentTok{# LRT then should be approximately Chi-squared distributed with df equal to the number of fixed dimensions (difference in free parameters between the full and reduced model).}

\CommentTok{# Likelihood-Ratio test (frequentist)}
\NormalTok{Deviance <-}\StringTok{ }\NormalTok{loglk_bic_reduced }\OperatorTok{-}\StringTok{ }\NormalTok{loglk_bic_full }
\NormalTok{Deviance}
\end{Highlighting}
\end{Shaded}

\begin{verbatim}
## [1] -74.746
\end{verbatim}

\begin{Shaded}
\begin{Highlighting}[]
\NormalTok{Chisq.crit <-}\StringTok{ }\KeywordTok{qchisq}\NormalTok{(}\FloatTok{0.95}\NormalTok{,}\DecValTok{1}\NormalTok{)}
\NormalTok{Chisq.crit}
\end{Highlighting}
\end{Shaded}

\begin{verbatim}
## [1] 3.841459
\end{verbatim}

\begin{Shaded}
\begin{Highlighting}[]
\CommentTok{# LRT}
\NormalTok{Deviance }\OperatorTok{>=}\StringTok{ }\NormalTok{Chisq.crit   }\CommentTok{# perform the LRT}
\end{Highlighting}
\end{Shaded}

\begin{verbatim}
## [1] FALSE
\end{verbatim}

\begin{Shaded}
\begin{Highlighting}[]
\CommentTok{# p-value}
\DecValTok{1}\OperatorTok{-}\KeywordTok{pchisq}\NormalTok{(Deviance,}\DecValTok{1}\NormalTok{)}
\end{Highlighting}
\end{Shaded}

\begin{verbatim}
## [1] 1
\end{verbatim}

\hypertarget{problem-3-solution}{%
\subsubsection{Problem 3 -- Solution}\label{problem-3-solution}}

\begin{enumerate}
\def\labelenumi{(\alph{enumi})}
\item
\end{enumerate}

\begin{Shaded}
\begin{Highlighting}[]
\KeywordTok{library}\NormalTok{(MASS)}
\KeywordTok{library}\NormalTok{(caret)}
\end{Highlighting}
\end{Shaded}

\begin{verbatim}
## Loading required package: lattice
\end{verbatim}

\begin{verbatim}
## 
## Attaching package: 'caret'
\end{verbatim}

\begin{verbatim}
## The following object is masked from 'package:purrr':
## 
##     lift
\end{verbatim}

\begin{Shaded}
\begin{Highlighting}[]
\KeywordTok{data}\NormalTok{(Boston)}

\NormalTok{tr.ind =}\StringTok{ }\DecValTok{1}\OperatorTok{:}\DecValTok{350}

\CommentTok{# Train & Test dataset}
\NormalTok{train <-}\StringTok{ }\NormalTok{Boston[tr.ind, ]}
\NormalTok{test <-}\StringTok{ }\NormalTok{Boston[}\OperatorTok{-}\NormalTok{tr.ind, ]}

\CommentTok{# Using all variables as predictors}
\NormalTok{train_lm.fit <-}\StringTok{ }\KeywordTok{lm}\NormalTok{(medv }\OperatorTok{~}\StringTok{ }\NormalTok{., train)}
\NormalTok{test_lm.fit <-}\StringTok{ }\KeywordTok{lm}\NormalTok{(medv }\OperatorTok{~}\StringTok{ }\NormalTok{., test)}

\CommentTok{# MSE for train and test data based models}
\NormalTok{mse_train <-}\StringTok{ }\KeywordTok{sum}\NormalTok{( train_lm.fit}\OperatorTok{$}\NormalTok{residuals}\OperatorTok{^}\DecValTok{2}\NormalTok{ )}\OperatorTok{/}\NormalTok{train_lm.fit}\OperatorTok{$}\NormalTok{df.residual}
\NormalTok{mse_train}
\end{Highlighting}
\end{Shaded}

\begin{verbatim}
## [1] 9.361612
\end{verbatim}

\begin{Shaded}
\begin{Highlighting}[]
\NormalTok{mse_test <-}\StringTok{ }\KeywordTok{sum}\NormalTok{( test_lm.fit}\OperatorTok{$}\NormalTok{residuals}\OperatorTok{^}\DecValTok{2}\NormalTok{ )}\OperatorTok{/}\NormalTok{test_lm.fit}\OperatorTok{$}\NormalTok{df.residual}
\NormalTok{mse_test}
\end{Highlighting}
\end{Shaded}

\begin{verbatim}
## [1] 23.09557
\end{verbatim}

\hypertarget{the-test-data-model-has-a-higher-rmse-than-the-train-dataset.-this-could-be-an-indication-that-your-model-is-overfitting}{%
\paragraph{The test data model has a higher RMSE than the train dataset.
This could be an indication that your model is
overfitting}\label{the-test-data-model-has-a-higher-rmse-than-the-train-dataset.-this-could-be-an-indication-that-your-model-is-overfitting}}

\begin{enumerate}
\def\labelenumi{(\alph{enumi})}
\setcounter{enumi}{1}
\item
\end{enumerate}

\begin{Shaded}
\begin{Highlighting}[]
\CommentTok{# 10-fold cross-validation (CV) within training data}
\NormalTok{modelcv <-}\StringTok{ }\KeywordTok{train}\NormalTok{(}
\NormalTok{  medv }\OperatorTok{~}\StringTok{ }\NormalTok{.,}
  \DataTypeTok{data =}\NormalTok{ train,}
  \DataTypeTok{method =} \StringTok{"lm"}\NormalTok{,}
  \DataTypeTok{trControl =} \KeywordTok{trainControl}\NormalTok{(}
    \DataTypeTok{method =} \StringTok{"cv"}\NormalTok{, }\DataTypeTok{number =} \DecValTok{10}
\NormalTok{  )}
\NormalTok{)}

\NormalTok{RMSE_Modelcv <-}\StringTok{ }\NormalTok{modelcv}\OperatorTok{$}\NormalTok{results}\OperatorTok{$}\NormalTok{RMSE}
\NormalTok{RMSE_Modelcv}
\end{Highlighting}
\end{Shaded}

\begin{verbatim}
## [1] 3.132821
\end{verbatim}

\begin{Shaded}
\begin{Highlighting}[]
\NormalTok{pcv <-}\StringTok{ }\KeywordTok{predict}\NormalTok{(modelcv, test)}
\NormalTok{errorcv <-}\StringTok{ }\NormalTok{(pcv}\OperatorTok{-}\StringTok{ }\NormalTok{test}\OperatorTok{$}\NormalTok{medv)}
\NormalTok{RMSE_NewDatacv <-}\StringTok{ }\KeywordTok{sqrt}\NormalTok{(}\KeywordTok{mean}\NormalTok{(errorcv}\OperatorTok{^}\DecValTok{2}\NormalTok{))}
\NormalTok{RMSE_NewDatacv}
\end{Highlighting}
\end{Shaded}

\begin{verbatim}
## [1] 23.36543
\end{verbatim}

\hypertarget{the-cross-validation-didnt-work-that-well-at-all.-it-could-still-be-the-case-of-overfitting-because-in-k-fold-cross-validation-the-dataset-is-divided-into-k-separate-parts.-the-training-process-is-repeated-k-times.-each-time-one-part-is-used-as-validation-data-and-the-rest-is-used-for-training-a-model.-and-in-this-we-arent-randomly-doing-that-which-increases-overfitting-error}{%
\paragraph{THe cross validation didnt work that well at all. It could
still be the case of overfitting because in K-fold cross validation, the
dataset is divided into k separate parts. The training process is
repeated k times. Each time, one part is used as validation data, and
the rest is used for training a model. ANd in this, we arent randomly
doing that which increases overfitting
error}\label{the-cross-validation-didnt-work-that-well-at-all.-it-could-still-be-the-case-of-overfitting-because-in-k-fold-cross-validation-the-dataset-is-divided-into-k-separate-parts.-the-training-process-is-repeated-k-times.-each-time-one-part-is-used-as-validation-data-and-the-rest-is-used-for-training-a-model.-and-in-this-we-arent-randomly-doing-that-which-increases-overfitting-error}}

\begin{enumerate}
\def\labelenumi{(\alph{enumi})}
\setcounter{enumi}{2}
\item
\end{enumerate}

\begin{Shaded}
\begin{Highlighting}[]
\NormalTok{tr.ind <-}\StringTok{ }\KeywordTok{read.csv}\NormalTok{(}\StringTok{"E:/Dropbox/Important_Documents/Doctoral_Work/Courses/High Dimensional Stats/2019/Week 3/HDS_ex3.3_tr.txt"}\NormalTok{, }\DataTypeTok{sep=}\StringTok{" "}\NormalTok{)}

\CommentTok{# Train & Test dataset}

\NormalTok{id <-}\StringTok{ }\KeywordTok{which}\NormalTok{(}\KeywordTok{rownames}\NormalTok{(Boston) }\OperatorTok\StringTok{ }\NormalTok{tr.ind}\OperatorTok{$}\NormalTok{V1 )}
\NormalTok{train <-}\StringTok{ }\NormalTok{Boston[id, ]}
\NormalTok{test <-}\StringTok{ }\NormalTok{Boston[}\OperatorTok{-}\NormalTok{id, ]}

\CommentTok{# Using all variables as predictors}
\NormalTok{train_lm.fit <-}\StringTok{ }\KeywordTok{lm}\NormalTok{(medv }\OperatorTok{~}\StringTok{ }\NormalTok{., train)}
\end{Highlighting}
\end{Shaded}

\begin{verbatim}
## Error in lm.fit(x, y, offset = offset, singular.ok = singular.ok, ...): 0 (non-NA) cases
\end{verbatim}

\begin{Shaded}
\begin{Highlighting}[]
\NormalTok{test_lm.fit <-}\StringTok{ }\KeywordTok{lm}\NormalTok{(medv }\OperatorTok{~}\StringTok{ }\NormalTok{., test)}
\end{Highlighting}
\end{Shaded}

\begin{verbatim}
## Error in lm.fit(x, y, offset = offset, singular.ok = singular.ok, ...): 0 (non-NA) cases
\end{verbatim}

\begin{Shaded}
\begin{Highlighting}[]
\CommentTok{# MSE for train and test data based models}
\NormalTok{mse_train <-}\StringTok{ }\KeywordTok{sum}\NormalTok{( train_lm.fit}\OperatorTok{$}\NormalTok{residuals}\OperatorTok{^}\DecValTok{2}\NormalTok{ )}\OperatorTok{/}\NormalTok{train_lm.fit}\OperatorTok{$}\NormalTok{df.residual}
\NormalTok{mse_train}
\end{Highlighting}
\end{Shaded}

\begin{verbatim}
## [1] 9.361612
\end{verbatim}

\begin{Shaded}
\begin{Highlighting}[]
\NormalTok{mse_test <-}\StringTok{ }\KeywordTok{sum}\NormalTok{( test_lm.fit}\OperatorTok{$}\NormalTok{residuals}\OperatorTok{^}\DecValTok{2}\NormalTok{ )}\OperatorTok{/}\NormalTok{test_lm.fit}\OperatorTok{$}\NormalTok{df.residual}
\NormalTok{mse_test}
\end{Highlighting}
\end{Shaded}

\begin{verbatim}
## [1] 23.09557
\end{verbatim}

\begin{Shaded}
\begin{Highlighting}[]
\CommentTok{# 10-fold cross-validation (CV) within training data}
\NormalTok{modelcv <-}\StringTok{ }\KeywordTok{train}\NormalTok{(}
\NormalTok{  medv }\OperatorTok{~}\StringTok{ }\NormalTok{.,}
  \DataTypeTok{data =}\NormalTok{ train,}
  \DataTypeTok{method =} \StringTok{"lm"}\NormalTok{,}
  \DataTypeTok{trControl =} \KeywordTok{trainControl}\NormalTok{(}
    \DataTypeTok{method =} \StringTok{"cv"}\NormalTok{, }\DataTypeTok{number =} \DecValTok{10}
\NormalTok{  )}
\NormalTok{)}
\end{Highlighting}
\end{Shaded}

\begin{verbatim}
## Error: Every row has at least one missing value were found
\end{verbatim}

\begin{Shaded}
\begin{Highlighting}[]
\NormalTok{RMSE_Modelcv <-}\StringTok{ }\NormalTok{modelcv}\OperatorTok{$}\NormalTok{results}\OperatorTok{$}\NormalTok{RMSE}
\NormalTok{RMSE_Modelcv}
\end{Highlighting}
\end{Shaded}

\begin{verbatim}
## [1] 3.132821
\end{verbatim}

\begin{Shaded}
\begin{Highlighting}[]
\NormalTok{pcv <-}\StringTok{ }\KeywordTok{predict}\NormalTok{(modelcv, test)}
\NormalTok{errorcv <-}\StringTok{ }\NormalTok{(pcv}\OperatorTok{-}\StringTok{ }\NormalTok{test}\OperatorTok{$}\NormalTok{medv)}
\NormalTok{RMSE_NewDatacv <-}\StringTok{ }\KeywordTok{sqrt}\NormalTok{(}\KeywordTok{mean}\NormalTok{(errorcv}\OperatorTok{^}\DecValTok{2}\NormalTok{))}
\NormalTok{RMSE_NewDatacv}
\end{Highlighting}
\end{Shaded}

\begin{verbatim}
## [1] NaN
\end{verbatim}

\hypertarget{we-see-that-randomly-splitting-did-massively-improve-th-cv-based-rmse-because-it-remuxf3ved-underlying-bias-associated-in-separating-our-datasets-non-randomly-thus-restring-the-assumption-of-iid-for-inference.}{%
\section{We see that randomly splitting did massively improve th CV
based rmse because it remóved underlying bias associated in separating
our datasets non-randomly, thus restring the assumption of iid for
inference.}\label{we-see-that-randomly-splitting-did-massively-improve-th-cv-based-rmse-because-it-remuxf3ved-underlying-bias-associated-in-separating-our-datasets-non-randomly-thus-restring-the-assumption-of-iid-for-inference.}}

\hypertarget{problem-4-solution}{%
\subsubsection{Problem 4 -- Solution}\label{problem-4-solution}}

\begin{enumerate}
\def\labelenumi{(\alph{enumi})}
\item
\end{enumerate}

\begin{Shaded}
\begin{Highlighting}[]
\NormalTok{tr.ind <-}\StringTok{ }\KeywordTok{fread}\NormalTok{(}\StringTok{"E:/Dropbox/Important_Documents/Doctoral_Work/Courses/High Dimensional Stats/2019/Week 3/HDS_ex3.3_tr.txt"}\NormalTok{, }\DataTypeTok{sep=}\StringTok{" "}\NormalTok{)}
\end{Highlighting}
\end{Shaded}

\begin{verbatim}
## Error in fread("E:/Dropbox/Important_Documents/Doctoral_Work/Courses/High Dimensional Stats/2019/Week 3/HDS_ex3.3_tr.txt", : could not find function "fread"
\end{verbatim}

\begin{Shaded}
\begin{Highlighting}[]
\CommentTok{# Train & Test dataset}
\NormalTok{train <-}\StringTok{ }\NormalTok{Boston[tr.ind}\OperatorTok{$}\NormalTok{V1, ]}

\CommentTok{# Create null model}
\NormalTok{glm.null <-}\StringTok{ }\KeywordTok{glm}\NormalTok{(medv }\OperatorTok{~}\StringTok{ }\DecValTok{1}\NormalTok{, }\DataTypeTok{data =}\NormalTok{ train)}
\end{Highlighting}
\end{Shaded}

\begin{verbatim}
## Warning in glm.fit(x = structure(numeric(0), .Dim = 0:1, .Dimnames = list(:
## no observations informative at iteration 1
\end{verbatim}

\begin{verbatim}
## Warning: glm.fit: algorithm did not converge
\end{verbatim}

\begin{verbatim}
## Error in glm.fit(x = structure(numeric(0), .Dim = 0:1, .Dimnames = list(: object 'fit' not found
\end{verbatim}

\begin{Shaded}
\begin{Highlighting}[]
\CommentTok{# Using all variables as predictors}
\NormalTok{train_glm.fit <-}\StringTok{ }\KeywordTok{glm}\NormalTok{(medv }\OperatorTok{~}\StringTok{ }\NormalTok{., }\DataTypeTok{family =}\NormalTok{ gaussian, train)}
\end{Highlighting}
\end{Shaded}

\begin{verbatim}
## Warning in glm.fit(x = structure(numeric(0), .Dim = c(0L, 14L), .Dimnames =
## list(: no observations informative at iteration 1

## Warning in glm.fit(x = structure(numeric(0), .Dim = c(0L, 14L), .Dimnames =
## list(: glm.fit: algorithm did not converge
\end{verbatim}

\begin{verbatim}
## Error in glm.fit(x = structure(numeric(0), .Dim = c(0L, 14L), .Dimnames = list(: object 'fit' not found
\end{verbatim}

\begin{Shaded}
\begin{Highlighting}[]
\CommentTok{# AIC-based forward selection}
\NormalTok{model.aic.forward <-}\StringTok{ }\KeywordTok{step}\NormalTok{(glm.null, }\DataTypeTok{direction =} \StringTok{"forward"}\NormalTok{, }\DataTypeTok{k =} \DecValTok{2}\NormalTok{, }\DataTypeTok{trace =} \OtherTok{FALSE}\NormalTok{,}
                          \DataTypeTok{scope =} \KeywordTok{list}\NormalTok{(}\DataTypeTok{lower=}\NormalTok{glm.null, }\DataTypeTok{upper=}\NormalTok{train_glm.fit))}
\end{Highlighting}
\end{Shaded}

\begin{verbatim}
## Error in terms(object): object 'glm.null' not found
\end{verbatim}

\begin{Shaded}
\begin{Highlighting}[]
\KeywordTok{summary}\NormalTok{(model.aic.forward)}
\end{Highlighting}
\end{Shaded}

\begin{verbatim}
## Error in summary(model.aic.forward): object 'model.aic.forward' not found
\end{verbatim}

\begin{Shaded}
\begin{Highlighting}[]
\CommentTok{# BIC-based forward selection}
\NormalTok{model.bic.forward <-}\StringTok{ }\KeywordTok{step}\NormalTok{(glm.null, }\DataTypeTok{direction =} \StringTok{"forward"}\NormalTok{, }\DataTypeTok{k =} \KeywordTok{log}\NormalTok{(}\KeywordTok{nrow}\NormalTok{(train)), }\DataTypeTok{trace =} \OtherTok{FALSE}\NormalTok{,}
                          \DataTypeTok{scope =} \KeywordTok{list}\NormalTok{(}\DataTypeTok{lower=}\NormalTok{glm.null, }\DataTypeTok{upper=}\NormalTok{train_glm.fit))}
\end{Highlighting}
\end{Shaded}

\begin{verbatim}
## Error in terms(object): object 'glm.null' not found
\end{verbatim}

\begin{Shaded}
\begin{Highlighting}[]
\KeywordTok{summary}\NormalTok{(model.bic.forward)}
\end{Highlighting}
\end{Shaded}

\begin{verbatim}
## Error in summary(model.bic.forward): object 'model.bic.forward' not found
\end{verbatim}

\begin{enumerate}
\def\labelenumi{(\alph{enumi})}
\setcounter{enumi}{1}
\item
\end{enumerate}

\begin{Shaded}
\begin{Highlighting}[]
\CommentTok{# AIC-based backward selection}
\NormalTok{model.aic.backward <-}\StringTok{ }\KeywordTok{step}\NormalTok{(train_glm.fit,  }\DataTypeTok{data =}\NormalTok{ train, }\DataTypeTok{direction =} \StringTok{"backward"}\NormalTok{, }\DataTypeTok{k =} \DecValTok{2}\NormalTok{, }\DataTypeTok{trace =} \OtherTok{FALSE}\NormalTok{)}
\end{Highlighting}
\end{Shaded}

\begin{verbatim}
## Error in terms(object): object 'train_glm.fit' not found
\end{verbatim}

\begin{Shaded}
\begin{Highlighting}[]
\KeywordTok{summary}\NormalTok{(model.aic.backward)}
\end{Highlighting}
\end{Shaded}

\begin{verbatim}
## Error in summary(model.aic.backward): object 'model.aic.backward' not found
\end{verbatim}

\begin{Shaded}
\begin{Highlighting}[]
\CommentTok{# BIC-based backward selection}
\NormalTok{model.bic.backward <-}\StringTok{ }\KeywordTok{step}\NormalTok{(train_glm.fit, }\DataTypeTok{direction =} \StringTok{"backward"}\NormalTok{, }\DataTypeTok{k =} \KeywordTok{log}\NormalTok{(}\KeywordTok{nrow}\NormalTok{(train)), }\DataTypeTok{trace =} \OtherTok{FALSE}\NormalTok{)}
\end{Highlighting}
\end{Shaded}

\begin{verbatim}
## Error in terms(object): object 'train_glm.fit' not found
\end{verbatim}

\begin{Shaded}
\begin{Highlighting}[]
\KeywordTok{summary}\NormalTok{(model.bic.backward)}
\end{Highlighting}
\end{Shaded}

\begin{verbatim}
## Error in summary(model.bic.backward): object 'model.bic.backward' not found
\end{verbatim}

\begin{enumerate}
\def\labelenumi{(\alph{enumi})}
\setcounter{enumi}{2}
\item
\end{enumerate}

\begin{Shaded}
\begin{Highlighting}[]
\CommentTok{# Use AIC for interaction terms in the model}
\CommentTok{# Limiting to 2nd order interactions only}

\CommentTok{# AIC-based forward selection}
\NormalTok{model.aic.interaction.forward <-}\StringTok{ }\KeywordTok{step}\NormalTok{(train_glm.fit, }\DataTypeTok{direction =} \StringTok{"forward"}\NormalTok{, }\DataTypeTok{k =} \DecValTok{2}\NormalTok{, }\DataTypeTok{trace =} \OtherTok{FALSE}\NormalTok{,}
                                      \DataTypeTok{scope =}\NormalTok{ . }\OperatorTok{~}\StringTok{ }\NormalTok{.}\OperatorTok{^}\DecValTok{2}\NormalTok{)}
\end{Highlighting}
\end{Shaded}

\begin{verbatim}
## Error in terms(object): object 'train_glm.fit' not found
\end{verbatim}

\begin{Shaded}
\begin{Highlighting}[]
\KeywordTok{summary}\NormalTok{(model.aic.interaction.forward)}
\end{Highlighting}
\end{Shaded}

\begin{verbatim}
## Error in summary(model.aic.interaction.forward): object 'model.aic.interaction.forward' not found
\end{verbatim}

\begin{Shaded}
\begin{Highlighting}[]
\CommentTok{# BIC-based forward selection}
\NormalTok{model.bic.interaction.forward <-}\StringTok{ }\KeywordTok{step}\NormalTok{(train_glm.fit, }\DataTypeTok{direction =} \StringTok{"forward"}\NormalTok{, }\DataTypeTok{k =} \KeywordTok{log}\NormalTok{(}\KeywordTok{nrow}\NormalTok{(train)), }
                                      \DataTypeTok{trace =} \OtherTok{FALSE}\NormalTok{, }\DataTypeTok{scope =}\NormalTok{ . }\OperatorTok{~}\StringTok{ }\NormalTok{.}\OperatorTok{^}\DecValTok{2}\NormalTok{)}
\end{Highlighting}
\end{Shaded}

\begin{verbatim}
## Error in terms(object): object 'train_glm.fit' not found
\end{verbatim}

\begin{Shaded}
\begin{Highlighting}[]
\KeywordTok{summary}\NormalTok{(model.bic.interaction.forward)}
\end{Highlighting}
\end{Shaded}

\begin{verbatim}
## Error in summary(model.bic.interaction.forward): object 'model.bic.interaction.forward' not found
\end{verbatim}

\begin{enumerate}
\def\labelenumi{(\alph{enumi})}
\setcounter{enumi}{3}
\item
\end{enumerate}

\begin{Shaded}
\begin{Highlighting}[]
\CommentTok{##################### Getting predictors + MSE from previous models}

\CommentTok{# Use glm on test data}
\NormalTok{test_glm.fit <-}\StringTok{ }\KeywordTok{glm}\NormalTok{(medv }\OperatorTok{~}\StringTok{ }\NormalTok{., }\DataTypeTok{family =}\NormalTok{ gaussian, test)}
\end{Highlighting}
\end{Shaded}

\begin{verbatim}
## Warning in glm.fit(x = structure(numeric(0), .Dim = c(0L, 14L), .Dimnames =
## list(: no observations informative at iteration 1
\end{verbatim}

\begin{verbatim}
## Warning: glm.fit: algorithm did not converge
\end{verbatim}

\begin{verbatim}
## Error in glm.fit(x = structure(numeric(0), .Dim = c(0L, 14L), .Dimnames = list(: object 'fit' not found
\end{verbatim}

\begin{Shaded}
\begin{Highlighting}[]
\CommentTok{##### AIC_forward model}
\NormalTok{AIC_forward <-}\StringTok{ }\KeywordTok{attr}\NormalTok{(model.aic.forward}\OperatorTok{$}\NormalTok{terms , }\StringTok{"term.labels"}\NormalTok{)}
\end{Highlighting}
\end{Shaded}

\begin{verbatim}
## Error in eval(expr, envir, enclos): object 'model.aic.forward' not found
\end{verbatim}

\begin{Shaded}
\begin{Highlighting}[]
\NormalTok{AIC_forward}
\end{Highlighting}
\end{Shaded}

\begin{verbatim}
## Error in eval(expr, envir, enclos): object 'AIC_forward' not found
\end{verbatim}

\begin{Shaded}
\begin{Highlighting}[]
\CommentTok{# Train}
\NormalTok{model.aic.forward.train <-}\StringTok{ }\KeywordTok{step}\NormalTok{(glm.null, }\DataTypeTok{data =}\NormalTok{ train, }\DataTypeTok{direction =} \StringTok{"forward"}\NormalTok{, }\DataTypeTok{k =} \DecValTok{2}\NormalTok{, }\DataTypeTok{trace =} \OtherTok{FALSE}\NormalTok{,}
                          \DataTypeTok{scope =} \KeywordTok{list}\NormalTok{(}\DataTypeTok{lower=}\NormalTok{glm.null, }\DataTypeTok{upper=}\NormalTok{train_glm.fit))}
\end{Highlighting}
\end{Shaded}

\begin{verbatim}
## Error in terms(object): object 'glm.null' not found
\end{verbatim}

\begin{Shaded}
\begin{Highlighting}[]
\NormalTok{AIC_forward_mse_train <-}\StringTok{ }\KeywordTok{sum}\NormalTok{( model.aic.forward.train}\OperatorTok{$}\NormalTok{residuals}\OperatorTok{^}\DecValTok{2}\NormalTok{ )}\OperatorTok{/}\NormalTok{model.aic.forward.train}\OperatorTok{$}\NormalTok{df.residual}
\end{Highlighting}
\end{Shaded}

\begin{verbatim}
## Error in eval(expr, envir, enclos): object 'model.aic.forward.train' not found
\end{verbatim}

\begin{Shaded}
\begin{Highlighting}[]
\CommentTok{# Test}
\NormalTok{model.aic.forward.test <-}\StringTok{ }\KeywordTok{step}\NormalTok{(glm.null, }\DataTypeTok{data =}\NormalTok{ test, }\DataTypeTok{direction =} \StringTok{"forward"}\NormalTok{, }\DataTypeTok{k =} \DecValTok{2}\NormalTok{, }\DataTypeTok{trace =} \OtherTok{FALSE}\NormalTok{,}
                          \DataTypeTok{scope =} \KeywordTok{list}\NormalTok{(}\DataTypeTok{lower=}\NormalTok{glm.null, }\DataTypeTok{upper=}\NormalTok{test_glm.fit))}
\end{Highlighting}
\end{Shaded}

\begin{verbatim}
## Error in terms(object): object 'glm.null' not found
\end{verbatim}

\begin{Shaded}
\begin{Highlighting}[]
\NormalTok{AIC_forward_mse_test <-}\StringTok{ }\KeywordTok{sum}\NormalTok{( model.aic.forward.test}\OperatorTok{$}\NormalTok{residuals}\OperatorTok{^}\DecValTok{2}\NormalTok{ )}\OperatorTok{/}\NormalTok{model.aic.forward.test}\OperatorTok{$}\NormalTok{df.residual}
\end{Highlighting}
\end{Shaded}

\begin{verbatim}
## Error in eval(expr, envir, enclos): object 'model.aic.forward.test' not found
\end{verbatim}

\begin{Shaded}
\begin{Highlighting}[]
\CommentTok{##### BIC_forward model}
\NormalTok{BIC_forward <-}\StringTok{ }\KeywordTok{attr}\NormalTok{(model.bic.forward}\OperatorTok{$}\NormalTok{terms , }\StringTok{"term.labels"}\NormalTok{)}
\end{Highlighting}
\end{Shaded}

\begin{verbatim}
## Error in eval(expr, envir, enclos): object 'model.bic.forward' not found
\end{verbatim}

\begin{Shaded}
\begin{Highlighting}[]
\NormalTok{BIC_forward}
\end{Highlighting}
\end{Shaded}

\begin{verbatim}
## Error in eval(expr, envir, enclos): object 'BIC_forward' not found
\end{verbatim}

\begin{Shaded}
\begin{Highlighting}[]
\CommentTok{# Train}
\NormalTok{model.bic.forward.train <-}\StringTok{ }\KeywordTok{step}\NormalTok{(glm.null, }\DataTypeTok{data =}\NormalTok{ train, }\DataTypeTok{direction =} \StringTok{"forward"}\NormalTok{, }\DataTypeTok{k =} \KeywordTok{log}\NormalTok{(}\KeywordTok{nrow}\NormalTok{(train)), }\DataTypeTok{trace =} \OtherTok{FALSE}\NormalTok{,}
                          \DataTypeTok{scope =} \KeywordTok{list}\NormalTok{(}\DataTypeTok{lower=}\NormalTok{glm.null, }\DataTypeTok{upper=}\NormalTok{train_glm.fit))}
\end{Highlighting}
\end{Shaded}

\begin{verbatim}
## Error in terms(object): object 'glm.null' not found
\end{verbatim}

\begin{Shaded}
\begin{Highlighting}[]
\NormalTok{BIC_forward_mse_train <-}\StringTok{ }\KeywordTok{sum}\NormalTok{( model.bic.forward.train}\OperatorTok{$}\NormalTok{residuals}\OperatorTok{^}\DecValTok{2}\NormalTok{ )}\OperatorTok{/}\NormalTok{model.bic.forward.train}\OperatorTok{$}\NormalTok{df.residual}
\end{Highlighting}
\end{Shaded}

\begin{verbatim}
## Error in eval(expr, envir, enclos): object 'model.bic.forward.train' not found
\end{verbatim}

\begin{Shaded}
\begin{Highlighting}[]
\CommentTok{# Test}
\NormalTok{model.bic.forward.test <-}\StringTok{ }\KeywordTok{step}\NormalTok{(glm.null, }\DataTypeTok{data =}\NormalTok{ test, }\DataTypeTok{direction =} \StringTok{"forward"}\NormalTok{, }\DataTypeTok{k =} \KeywordTok{log}\NormalTok{(}\KeywordTok{nrow}\NormalTok{(test)), }\DataTypeTok{trace =} \OtherTok{FALSE}\NormalTok{,}
                          \DataTypeTok{scope =} \KeywordTok{list}\NormalTok{(}\DataTypeTok{lower=}\NormalTok{glm.null, }\DataTypeTok{upper=}\NormalTok{test_glm.fit))}
\end{Highlighting}
\end{Shaded}

\begin{verbatim}
## Error in terms(object): object 'glm.null' not found
\end{verbatim}

\begin{Shaded}
\begin{Highlighting}[]
\NormalTok{BIC_forward_mse_test <-}\StringTok{ }\KeywordTok{sum}\NormalTok{( model.bic.forward.test}\OperatorTok{$}\NormalTok{residuals}\OperatorTok{^}\DecValTok{2}\NormalTok{ )}\OperatorTok{/}\NormalTok{model.bic.forward.test}\OperatorTok{$}\NormalTok{df.residual}
\end{Highlighting}
\end{Shaded}

\begin{verbatim}
## Error in eval(expr, envir, enclos): object 'model.bic.forward.test' not found
\end{verbatim}

\begin{Shaded}
\begin{Highlighting}[]
\CommentTok{##### AIC_backward model}
\NormalTok{AIC_backward <-}\StringTok{ }\KeywordTok{attr}\NormalTok{(model.aic.backward}\OperatorTok{$}\NormalTok{terms , }\StringTok{"term.labels"}\NormalTok{)}
\end{Highlighting}
\end{Shaded}

\begin{verbatim}
## Error in eval(expr, envir, enclos): object 'model.aic.backward' not found
\end{verbatim}

\begin{Shaded}
\begin{Highlighting}[]
\NormalTok{AIC_backward}
\end{Highlighting}
\end{Shaded}

\begin{verbatim}
## Error in eval(expr, envir, enclos): object 'AIC_backward' not found
\end{verbatim}

\begin{Shaded}
\begin{Highlighting}[]
\CommentTok{# Train}
\NormalTok{model.aic.backward.train <-}\StringTok{ }\KeywordTok{step}\NormalTok{(train_glm.fit,  }\DataTypeTok{data =}\NormalTok{ train, }\DataTypeTok{direction =} \StringTok{"backward"}\NormalTok{, }\DataTypeTok{k =} \DecValTok{2}\NormalTok{, }\DataTypeTok{trace =} \OtherTok{FALSE}\NormalTok{)}
\end{Highlighting}
\end{Shaded}

\begin{verbatim}
## Error in terms(object): object 'train_glm.fit' not found
\end{verbatim}

\begin{Shaded}
\begin{Highlighting}[]
\NormalTok{AIC_backward_mse_train <-}\StringTok{ }\KeywordTok{sum}\NormalTok{( model.aic.backward.train}\OperatorTok{$}\NormalTok{residuals}\OperatorTok{^}\DecValTok{2}\NormalTok{ )}\OperatorTok{/}\NormalTok{model.aic.backward.train}\OperatorTok{$}\NormalTok{df.residual}
\end{Highlighting}
\end{Shaded}

\begin{verbatim}
## Error in eval(expr, envir, enclos): object 'model.aic.backward.train' not found
\end{verbatim}

\begin{Shaded}
\begin{Highlighting}[]
\CommentTok{# Test}
\NormalTok{model.aic.backward.test <-}\StringTok{ }\KeywordTok{step}\NormalTok{(test_glm.fit,  }\DataTypeTok{data =}\NormalTok{ test, }\DataTypeTok{direction =} \StringTok{"backward"}\NormalTok{, }\DataTypeTok{k =} \DecValTok{2}\NormalTok{, }\DataTypeTok{trace =} \OtherTok{FALSE}\NormalTok{)}
\end{Highlighting}
\end{Shaded}

\begin{verbatim}
## Error in terms(object): object 'test_glm.fit' not found
\end{verbatim}

\begin{Shaded}
\begin{Highlighting}[]
\NormalTok{AIC_backward_mse_test <-}\StringTok{ }\KeywordTok{sum}\NormalTok{( model.aic.backward.test}\OperatorTok{$}\NormalTok{residuals}\OperatorTok{^}\DecValTok{2}\NormalTok{ )}\OperatorTok{/}\NormalTok{model.aic.backward.test}\OperatorTok{$}\NormalTok{df.residual}
\end{Highlighting}
\end{Shaded}

\begin{verbatim}
## Error in eval(expr, envir, enclos): object 'model.aic.backward.test' not found
\end{verbatim}

\begin{Shaded}
\begin{Highlighting}[]
\CommentTok{##### BIC_backward model}
\NormalTok{BIC_backward <-}\StringTok{ }\KeywordTok{attr}\NormalTok{(model.bic.backward}\OperatorTok{$}\NormalTok{terms , }\StringTok{"term.labels"}\NormalTok{)}
\end{Highlighting}
\end{Shaded}

\begin{verbatim}
## Error in eval(expr, envir, enclos): object 'model.bic.backward' not found
\end{verbatim}

\begin{Shaded}
\begin{Highlighting}[]
\NormalTok{BIC_backward}
\end{Highlighting}
\end{Shaded}

\begin{verbatim}
## Error in eval(expr, envir, enclos): object 'BIC_backward' not found
\end{verbatim}

\begin{Shaded}
\begin{Highlighting}[]
\CommentTok{# Train}
\NormalTok{model.bic.backward.train <-}\StringTok{ }\KeywordTok{step}\NormalTok{(train_glm.fit, }\DataTypeTok{data =}\NormalTok{ train, }\DataTypeTok{direction =} \StringTok{"backward"}\NormalTok{, }\DataTypeTok{k =} \KeywordTok{log}\NormalTok{(}\KeywordTok{nrow}\NormalTok{(train)), }\DataTypeTok{trace =} \OtherTok{FALSE}\NormalTok{)}
\end{Highlighting}
\end{Shaded}

\begin{verbatim}
## Error in terms(object): object 'train_glm.fit' not found
\end{verbatim}

\begin{Shaded}
\begin{Highlighting}[]
\NormalTok{BIC_backward_mse_train <-}\StringTok{ }\KeywordTok{sum}\NormalTok{( model.bic.backward.train}\OperatorTok{$}\NormalTok{residuals}\OperatorTok{^}\DecValTok{2}\NormalTok{ )}\OperatorTok{/}\NormalTok{model.bic.backward.train}\OperatorTok{$}\NormalTok{df.residual}
\end{Highlighting}
\end{Shaded}

\begin{verbatim}
## Error in eval(expr, envir, enclos): object 'model.bic.backward.train' not found
\end{verbatim}

\begin{Shaded}
\begin{Highlighting}[]
\CommentTok{# Test}
\NormalTok{model.bic.backward.test <-}\StringTok{ }\KeywordTok{step}\NormalTok{(test_glm.fit, }\DataTypeTok{data =}\NormalTok{ test, }\DataTypeTok{direction =} \StringTok{"backward"}\NormalTok{, }\DataTypeTok{k =} \KeywordTok{log}\NormalTok{(}\KeywordTok{nrow}\NormalTok{(test)), }\DataTypeTok{trace =} \OtherTok{FALSE}\NormalTok{)}
\end{Highlighting}
\end{Shaded}

\begin{verbatim}
## Error in terms(object): object 'test_glm.fit' not found
\end{verbatim}

\begin{Shaded}
\begin{Highlighting}[]
\NormalTok{BIC_backward_mse_test <-}\StringTok{ }\KeywordTok{sum}\NormalTok{( model.bic.backward.test}\OperatorTok{$}\NormalTok{residuals}\OperatorTok{^}\DecValTok{2}\NormalTok{ )}\OperatorTok{/}\NormalTok{model.bic.backward.test}\OperatorTok{$}\NormalTok{df.residual}
\end{Highlighting}
\end{Shaded}

\begin{verbatim}
## Error in eval(expr, envir, enclos): object 'model.bic.backward.test' not found
\end{verbatim}

\begin{Shaded}
\begin{Highlighting}[]
\CommentTok{##### AIC-based forward selection + interaction}
\NormalTok{AIC_forward_interaction <-}\StringTok{ }\KeywordTok{attr}\NormalTok{(model.aic.interaction.forward}\OperatorTok{$}\NormalTok{terms , }\StringTok{"term.labels"}\NormalTok{)}
\end{Highlighting}
\end{Shaded}

\begin{verbatim}
## Error in eval(expr, envir, enclos): object 'model.aic.interaction.forward' not found
\end{verbatim}

\begin{Shaded}
\begin{Highlighting}[]
\NormalTok{AIC_forward_interaction}
\end{Highlighting}
\end{Shaded}

\begin{verbatim}
## Error in eval(expr, envir, enclos): object 'AIC_forward_interaction' not found
\end{verbatim}

\begin{Shaded}
\begin{Highlighting}[]
\CommentTok{# Train}
\NormalTok{model.aic.interaction.forward.train <-}\StringTok{ }\KeywordTok{step}\NormalTok{(train_glm.fit,  }\DataTypeTok{data =}\NormalTok{ train, }\DataTypeTok{direction =} \StringTok{"forward"}\NormalTok{, }\DataTypeTok{k =} \DecValTok{2}\NormalTok{, }\DataTypeTok{trace =} \OtherTok{FALSE}\NormalTok{,}
                                      \DataTypeTok{scope =}\NormalTok{ . }\OperatorTok{~}\StringTok{ }\NormalTok{.}\OperatorTok{^}\DecValTok{2}\NormalTok{)}
\end{Highlighting}
\end{Shaded}

\begin{verbatim}
## Error in terms(object): object 'train_glm.fit' not found
\end{verbatim}

\begin{Shaded}
\begin{Highlighting}[]
\NormalTok{AIC_forward_interaction_mse_train <-}\StringTok{ }\KeywordTok{sum}\NormalTok{( model.aic.interaction.forward.train}\OperatorTok{$}\NormalTok{residuals}\OperatorTok{^}\DecValTok{2}\NormalTok{ )}\OperatorTok{/}\NormalTok{model.aic.interaction.forward.train}\OperatorTok{$}\NormalTok{df.residual}
\end{Highlighting}
\end{Shaded}

\begin{verbatim}
## Error in eval(expr, envir, enclos): object 'model.aic.interaction.forward.train' not found
\end{verbatim}

\begin{Shaded}
\begin{Highlighting}[]
\CommentTok{# Test}
\NormalTok{model.aic.interaction.forward.test <-}\StringTok{ }\KeywordTok{step}\NormalTok{(test_glm.fit, }\DataTypeTok{data =}\NormalTok{ test, }\DataTypeTok{direction =} \StringTok{"forward"}\NormalTok{, }\DataTypeTok{k =} \DecValTok{2}\NormalTok{, }\DataTypeTok{trace =} \OtherTok{FALSE}\NormalTok{,}
                                      \DataTypeTok{scope =}\NormalTok{ . }\OperatorTok{~}\StringTok{ }\NormalTok{.}\OperatorTok{^}\DecValTok{2}\NormalTok{)}
\end{Highlighting}
\end{Shaded}

\begin{verbatim}
## Error in terms(object): object 'test_glm.fit' not found
\end{verbatim}

\begin{Shaded}
\begin{Highlighting}[]
\NormalTok{AIC_forward_interaction_mse_test <-}\StringTok{ }\KeywordTok{sum}\NormalTok{( model.aic.interaction.forward.test}\OperatorTok{$}\NormalTok{residuals}\OperatorTok{^}\DecValTok{2}\NormalTok{ )}\OperatorTok{/}\NormalTok{model.aic.interaction.forward.test}\OperatorTok{$}\NormalTok{df.residual}
\end{Highlighting}
\end{Shaded}

\begin{verbatim}
## Error in eval(expr, envir, enclos): object 'model.aic.interaction.forward.test' not found
\end{verbatim}

\begin{Shaded}
\begin{Highlighting}[]
\CommentTok{##### BIC-based forward selection + interaction}
\NormalTok{BIC_forward_interaction <-}\StringTok{ }\KeywordTok{attr}\NormalTok{(model.bic.interaction.forward}\OperatorTok{$}\NormalTok{terms , }\StringTok{"term.labels"}\NormalTok{)}
\end{Highlighting}
\end{Shaded}

\begin{verbatim}
## Error in eval(expr, envir, enclos): object 'model.bic.interaction.forward' not found
\end{verbatim}

\begin{Shaded}
\begin{Highlighting}[]
\NormalTok{BIC_forward_interaction}
\end{Highlighting}
\end{Shaded}

\begin{verbatim}
## Error in eval(expr, envir, enclos): object 'BIC_forward_interaction' not found
\end{verbatim}

\begin{Shaded}
\begin{Highlighting}[]
\CommentTok{# Train}
\NormalTok{model.bic.interaction.forward.train <-}\StringTok{ }\KeywordTok{step}\NormalTok{(train_glm.fit, }\DataTypeTok{direction =} \StringTok{"forward"}\NormalTok{, }\DataTypeTok{k =} \KeywordTok{log}\NormalTok{(}\KeywordTok{nrow}\NormalTok{(train)), }
                                      \DataTypeTok{trace =} \OtherTok{FALSE}\NormalTok{, }\DataTypeTok{scope =}\NormalTok{ . }\OperatorTok{~}\StringTok{ }\NormalTok{.}\OperatorTok{^}\DecValTok{2}\NormalTok{)}
\end{Highlighting}
\end{Shaded}

\begin{verbatim}
## Error in terms(object): object 'train_glm.fit' not found
\end{verbatim}

\begin{Shaded}
\begin{Highlighting}[]
\NormalTok{BIC_forward_interaction_mse_train <-}\StringTok{ }\KeywordTok{sum}\NormalTok{( model.bic.interaction.forward}\OperatorTok{$}\NormalTok{residuals}\OperatorTok{^}\DecValTok{2}\NormalTok{ )}\OperatorTok{/}\NormalTok{model.bic.interaction.forward}\OperatorTok{$}\NormalTok{df.residual}
\end{Highlighting}
\end{Shaded}

\begin{verbatim}
## Error in eval(expr, envir, enclos): object 'model.bic.interaction.forward' not found
\end{verbatim}

\begin{Shaded}
\begin{Highlighting}[]
\CommentTok{# Test}
\NormalTok{model.bic.interaction.forward.test <-}\StringTok{ }\KeywordTok{step}\NormalTok{(train_glm.fit, }\DataTypeTok{direction =} \StringTok{"forward"}\NormalTok{, }\DataTypeTok{k =} \KeywordTok{log}\NormalTok{(}\KeywordTok{nrow}\NormalTok{(test)), }\DataTypeTok{trace =} \OtherTok{FALSE}\NormalTok{,}
                                      \DataTypeTok{scope =}\NormalTok{ . }\OperatorTok{~}\StringTok{ }\NormalTok{.}\OperatorTok{^}\DecValTok{2}\NormalTok{)}
\end{Highlighting}
\end{Shaded}

\begin{verbatim}
## Error in terms(object): object 'train_glm.fit' not found
\end{verbatim}

\begin{Shaded}
\begin{Highlighting}[]
\NormalTok{BIC_forward_interaction_mse_test <-}\StringTok{ }\KeywordTok{sum}\NormalTok{( model.bic.interaction.forward.test}\OperatorTok{$}\NormalTok{residuals}\OperatorTok{^}\DecValTok{2}\NormalTok{ )}\OperatorTok{/}\NormalTok{model.bic.interaction.forward.test}\OperatorTok{$}\NormalTok{df.residual}
\end{Highlighting}
\end{Shaded}

\begin{verbatim}
## Error in eval(expr, envir, enclos): object 'model.bic.interaction.forward.test' not found
\end{verbatim}

\begin{Shaded}
\begin{Highlighting}[]
\CommentTok{### Get results from above together}

\NormalTok{results <-}\StringTok{ }\KeywordTok{data.frame}\NormalTok{(}\KeywordTok{rbind}\NormalTok{( }
                  \KeywordTok{c}\NormalTok{(AIC_forward_mse_train, AIC_forward_mse_test),}
                  \KeywordTok{c}\NormalTok{(BIC_forward_mse_train, BIC_forward_mse_test),}
                  \KeywordTok{c}\NormalTok{(AIC_backward_mse_train, AIC_backward_mse_test),}
                  \KeywordTok{c}\NormalTok{(BIC_backward_mse_train, BIC_backward_mse_test),}
                  \KeywordTok{c}\NormalTok{(AIC_forward_interaction_mse_train, AIC_forward_interaction_mse_test),}
                  \KeywordTok{c}\NormalTok{(BIC_forward_interaction_mse_train, BIC_forward_interaction_mse_test)))}
\end{Highlighting}
\end{Shaded}

\begin{verbatim}
## Error in rbind(c(AIC_forward_mse_train, AIC_forward_mse_test), c(BIC_forward_mse_train, : object 'AIC_forward_mse_train' not found
\end{verbatim}

\begin{Shaded}
\begin{Highlighting}[]
\KeywordTok{colnames}\NormalTok{(results) <-}\StringTok{ }\KeywordTok{c}\NormalTok{(}\StringTok{"Training_MSE"}\NormalTok{, }\StringTok{"Test_MSE"}\NormalTok{)}
\NormalTok{results <-}\StringTok{ }\NormalTok{results }\OperatorTok\StringTok{ }\KeywordTok{mutate}\NormalTok{(}\DataTypeTok{Model =} \KeywordTok{c}\NormalTok{(}\StringTok{"AIC_forward"}\NormalTok{, }\StringTok{"BIC_forward"}\NormalTok{, }\StringTok{"AIC_backward"}\NormalTok{, }
                                        \StringTok{"BIC_backward"}\NormalTok{, }\StringTok{"AIC_forward_interaction"}\NormalTok{, }\StringTok{"BIC_forward_interaction"}\NormalTok{))}
\end{Highlighting}
\end{Shaded}

\begin{verbatim}
## Error: Columns 3, 4 must be named.
## Use .name_repair to specify repair.
\end{verbatim}

\begin{Shaded}
\begin{Highlighting}[]
\NormalTok{results}
\end{Highlighting}
\end{Shaded}

\begin{verbatim}
##   Training_MSE Test_MSE        NA        NA
## 1            1  0.00006   0.01160   0.00158
## 2            2  0.00028   0.00305   0.00650
## 3            9 25.35862 270.84226 246.74820
\end{verbatim}

\hypertarget{the-results-suggest-that-stepwise-aic-forward-regression-model-with-2nd-order-interaction-terms-has-the-lowest-rmse-indictaing-it-to-be-the-better-model.-the-variance-not-explained-is-given-by-msevariancey-and-hence-1-variance-unexplained-will-give-us-the-explained-variance}{%
\paragraph{The results suggest that stepwise AIC-forward regression
model with 2nd order interaction terms has the lowest RMSE indictaing it
to be the better model. The variance not explained is given by
MSE/Variance(Y) and hence 1-Variance Unexplained will give us the
explained
variance}\label{the-results-suggest-that-stepwise-aic-forward-regression-model-with-2nd-order-interaction-terms-has-the-lowest-rmse-indictaing-it-to-be-the-better-model.-the-variance-not-explained-is-given-by-msevariancey-and-hence-1-variance-unexplained-will-give-us-the-explained-variance}}

\begin{Shaded}
\begin{Highlighting}[]
\CommentTok{## Variance explained}
\NormalTok{(}\DecValTok{1} \OperatorTok{-}\StringTok{ }\NormalTok{AIC_forward_interaction_mse_test}\OperatorTok{/}\KeywordTok{var}\NormalTok{(test}\OperatorTok{$}\NormalTok{medv)) }\OperatorTok{*}\StringTok{ }\DecValTok{100}
\end{Highlighting}
\end{Shaded}

\begin{verbatim}
## Error in eval(expr, envir, enclos): object 'AIC_forward_interaction_mse_test' not found
\end{verbatim}

\hypertarget{problem-5-solution}{%
\subsubsection{Problem 5 -- Solution}\label{problem-5-solution}}

\begin{enumerate}
\def\labelenumi{(\alph{enumi})}
\item
\end{enumerate}

\begin{Shaded}
\begin{Highlighting}[]
\CommentTok{# Dataset}
\NormalTok{HDS_ex3 <-}\StringTok{ }\KeywordTok{fread}\NormalTok{(}\StringTok{"E:/Dropbox/Important_Documents/Doctoral_Work/Courses/High Dimensional Stats/2019/Week 3/HDS_ex3.txt"}\NormalTok{, }\DataTypeTok{sep=}\StringTok{" "}\NormalTok{)}
\end{Highlighting}
\end{Shaded}

\begin{verbatim}
## Error in fread("E:/Dropbox/Important_Documents/Doctoral_Work/Courses/High Dimensional Stats/2019/Week 3/HDS_ex3.txt", : could not find function "fread"
\end{verbatim}

\begin{Shaded}
\begin{Highlighting}[]
\CommentTok{# Null Model}
\NormalTok{glm.null <-}\StringTok{ }\KeywordTok{glm}\NormalTok{(z }\OperatorTok{~}\StringTok{ }\DecValTok{1}\NormalTok{, }\DataTypeTok{data =}\NormalTok{ HDS_ex3)}

\CommentTok{# Log Reg}
\NormalTok{glm.fit <-}\StringTok{ }\KeywordTok{glm}\NormalTok{(z }\OperatorTok{~}\StringTok{ }\NormalTok{x }\OperatorTok{+}\StringTok{ }\KeywordTok{I}\NormalTok{(x}\OperatorTok{^}\DecValTok{2}\NormalTok{), }\DataTypeTok{family =} \StringTok{"binomial"}\NormalTok{, }\DataTypeTok{data =}\NormalTok{ HDS_ex3)}

\CommentTok{# Forward stepwise selection with AIC}
\NormalTok{model.aic.forward <-}\StringTok{ }\KeywordTok{step}\NormalTok{(glm.null, }\DataTypeTok{data =}\NormalTok{ HDS_ex3, }\DataTypeTok{direction =} \StringTok{"forward"}\NormalTok{, }\DataTypeTok{k =} \DecValTok{2}\NormalTok{, }\DataTypeTok{trace =} \OtherTok{TRUE}\NormalTok{,}
                          \DataTypeTok{scope =} \KeywordTok{list}\NormalTok{(}\DataTypeTok{lower=}\NormalTok{glm.null, }\DataTypeTok{upper=}\NormalTok{glm.fit))}
\end{Highlighting}
\end{Shaded}

\begin{verbatim}
## Start:  AIC=162.36
## z ~ 1
## 
##          Df Deviance    AIC
## <none>        27.173 162.36
## + x       1   27.099 164.06
## + I(x^2)  1   27.118 164.13
\end{verbatim}

\begin{Shaded}
\begin{Highlighting}[]
\KeywordTok{summary}\NormalTok{(model.aic.forward)}
\end{Highlighting}
\end{Shaded}

\begin{verbatim}
## 
## Call:
## glm(formula = z ~ 1, data = HDS_ex3)
## 
## Deviance Residuals: 
##     Min       1Q   Median       3Q      Max  
## -0.5545  -0.5545   0.4455   0.4455   0.4455  
## 
## Coefficients:
##             Estimate Std. Error t value Pr(>|t|)    
## (Intercept)  0.55455    0.04761   11.65   <2e-16 ***
## ---
## Signif. codes:  0 '***' 0.001 '**' 0.01 '*' 0.05 '.' 0.1 ' ' 1
## 
## (Dispersion parameter for gaussian family taken to be 0.2492911)
## 
##     Null deviance: 27.173  on 109  degrees of freedom
## Residual deviance: 27.173  on 109  degrees of freedom
## AIC: 162.36
## 
## Number of Fisher Scoring iterations: 2
\end{verbatim}

\begin{Shaded}
\begin{Highlighting}[]
\CommentTok{### THe intercept model is the one chosen by the stepwise method}
\NormalTok{new.model <-}\StringTok{ }\KeywordTok{glm}\NormalTok{(z }\OperatorTok{~}\StringTok{ }\DecValTok{1} \OperatorTok{+}\StringTok{ }\NormalTok{x }\OperatorTok{+}\StringTok{ }\KeywordTok{I}\NormalTok{(x}\OperatorTok{^}\DecValTok{2}\NormalTok{), }\DataTypeTok{family =} \StringTok{"binomial"}\NormalTok{, }\DataTypeTok{data =}\NormalTok{ HDS_ex3)}
\KeywordTok{AIC}\NormalTok{(new.model)}
\end{Highlighting}
\end{Shaded}

\begin{verbatim}
## [1] 139.7259
\end{verbatim}

\hypertarget{we-may-not-necessarily-always-get-the-highest-variance-explained-and-lower-aic-because-we-only-compare-a-subset-of-possible-models-and-might-miss-the-one-with-the-highest-adjr2lowest-aic-which-would-include-all-the-variables-like-the-above-case.-given-a-set-of-predictors-there-is-no-guarantee-that-stepwise-will-find-the-best-combination-of-predictors-defined-as-say-the-highest-adjusted-r2-it-can-get-stuck-in-local-optimaand-never-reach-the-so-called-global-optima-which-might-be-the-desired-solution}{%
\paragraph{We may not necessarily always get the highest variance
explained and lower AIC because we only compare a subset of possible
models and might miss the one with the highest adjR2/lowest AIC which
would include all the variables, like the above case. Given a set of
predictors, there is no guarantee that stepwise will find the ``best''
combination of predictors (defined as, say, the highest adjusted
R\^{}2); it can get stuck in local optima´and never reach the so-called
global optima which might be the desired
solution}\label{we-may-not-necessarily-always-get-the-highest-variance-explained-and-lower-aic-because-we-only-compare-a-subset-of-possible-models-and-might-miss-the-one-with-the-highest-adjr2lowest-aic-which-would-include-all-the-variables-like-the-above-case.-given-a-set-of-predictors-there-is-no-guarantee-that-stepwise-will-find-the-best-combination-of-predictors-defined-as-say-the-highest-adjusted-r2-it-can-get-stuck-in-local-optimaand-never-reach-the-so-called-global-optima-which-might-be-the-desired-solution}}

\begin{enumerate}
\def\labelenumi{(\alph{enumi})}
\setcounter{enumi}{1}
\item
\end{enumerate}

\begin{Shaded}
\begin{Highlighting}[]
\CommentTok{# Logistic reg - 1}
\NormalTok{log.reg1 <-}\StringTok{ }\KeywordTok{glm}\NormalTok{(z }\OperatorTok{~}\StringTok{ }\DecValTok{1} \OperatorTok{+}\StringTok{ }\NormalTok{x }\OperatorTok{+}\StringTok{ }\KeywordTok{I}\NormalTok{(x}\OperatorTok{^}\DecValTok{2}\NormalTok{), }\DataTypeTok{family =} \StringTok{"binomial"}\NormalTok{, }\DataTypeTok{data =}\NormalTok{ HDS_ex3)}
\KeywordTok{summary}\NormalTok{(log.reg1)}
\end{Highlighting}
\end{Shaded}

\begin{verbatim}
## 
## Call:
## glm(formula = z ~ 1 + x + I(x^2), family = "binomial", data = HDS_ex3)
## 
## Deviance Residuals: 
##    Min      1Q  Median      3Q     Max  
## -2.130  -1.015   0.524   1.001   1.489  
## 
## Coefficients:
##             Estimate Std. Error z value Pr(>|z|)    
## (Intercept)   2.4880     0.7475   3.328 0.000873 ***
## x           -12.2659     3.3066  -3.709 0.000208 ***
## I(x^2)       11.7626     3.1477   3.737 0.000186 ***
## ---
## Signif. codes:  0 '***' 0.001 '**' 0.01 '*' 0.05 '.' 0.1 ' ' 1
## 
## (Dispersion parameter for binomial family taken to be 1)
## 
##     Null deviance: 151.18  on 109  degrees of freedom
## Residual deviance: 133.73  on 107  degrees of freedom
## AIC: 139.73
## 
## Number of Fisher Scoring iterations: 4
\end{verbatim}

\begin{Shaded}
\begin{Highlighting}[]
\CommentTok{# Logistic reg - 2}
\NormalTok{log.reg2 <-}\StringTok{ }\KeywordTok{glm}\NormalTok{(z }\OperatorTok{~}\StringTok{ }\DecValTok{1}\OperatorTok{+}\StringTok{ }\NormalTok{x }\OperatorTok{+}\StringTok{ }\KeywordTok{I}\NormalTok{(x}\OperatorTok{^}\DecValTok{2}\NormalTok{) }\OperatorTok{+}\StringTok{ }\NormalTok{y, }\DataTypeTok{family =} \StringTok{"binomial"}\NormalTok{, }\DataTypeTok{data =}\NormalTok{ HDS_ex3)}
\KeywordTok{summary}\NormalTok{(log.reg2)}
\end{Highlighting}
\end{Shaded}

\begin{verbatim}
## 
## Call:
## glm(formula = z ~ 1 + x + I(x^2) + y, family = "binomial", data = HDS_ex3)
## 
## Deviance Residuals: 
##     Min       1Q   Median       3Q      Max  
## -2.0623  -0.9985   0.4939   0.9803   1.8278  
## 
## Coefficients:
##             Estimate Std. Error z value Pr(>|z|)  
## (Intercept)   -6.267      4.396  -1.426   0.1540  
## x             -3.692      5.246  -0.704   0.4815  
## I(x^2)         3.540      5.026   0.704   0.4812  
## y              8.648      4.326   1.999   0.0456 *
## ---
## Signif. codes:  0 '***' 0.001 '**' 0.01 '*' 0.05 '.' 0.1 ' ' 1
## 
## (Dispersion parameter for binomial family taken to be 1)
## 
##     Null deviance: 151.18  on 109  degrees of freedom
## Residual deviance: 129.49  on 106  degrees of freedom
## AIC: 137.49
## 
## Number of Fisher Scoring iterations: 4
\end{verbatim}

\begin{enumerate}
\def\labelenumi{(\alph{enumi})}
\setcounter{enumi}{2}
\item
\end{enumerate}

\begin{Shaded}
\begin{Highlighting}[]
\CommentTok{# Backward Selection}
\NormalTok{model.aic.backward <-}\StringTok{ }\KeywordTok{step}\NormalTok{(log.reg2,  }\DataTypeTok{data =}\NormalTok{ HDS_ex3, }\DataTypeTok{direction =} \StringTok{"backward"}\NormalTok{, }\DataTypeTok{k =} \DecValTok{2}\NormalTok{, }\DataTypeTok{trace =} \OtherTok{FALSE}\NormalTok{)}
\KeywordTok{summary}\NormalTok{(model.aic.backward)}
\end{Highlighting}
\end{Shaded}

\begin{verbatim}
## 
## Call:
## glm(formula = z ~ y, family = "binomial", data = HDS_ex3)
## 
## Deviance Residuals: 
##     Min       1Q   Median       3Q      Max  
## -2.0331  -1.0155   0.5088   0.9903   1.9030  
## 
## Coefficients:
##             Estimate Std. Error z value Pr(>|z|)    
## (Intercept)   -8.961      2.262  -3.962 7.42e-05 ***
## y             11.065      2.737   4.043 5.28e-05 ***
## ---
## Signif. codes:  0 '***' 0.001 '**' 0.01 '*' 0.05 '.' 0.1 ' ' 1
## 
## (Dispersion parameter for binomial family taken to be 1)
## 
##     Null deviance: 151.18  on 109  degrees of freedom
## Residual deviance: 129.99  on 108  degrees of freedom
## AIC: 133.99
## 
## Number of Fisher Scoring iterations: 3
\end{verbatim}

\hypertarget{based-on-this-the-choice-is-the-model-with-just-y-formula-z-y-instead-of-x-and-its-quadratic-term.}{%
\paragraph{Based on this the choice is the model with just ``y''
(formula = z \textasciitilde{} y) instead of ``x'' and its quadratic
term.}\label{based-on-this-the-choice-is-the-model-with-just-y-formula-z-y-instead-of-x-and-its-quadratic-term.}}


\end{document}
